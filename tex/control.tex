\section{Control}
\label{sec:control}

\subsection{The KAT Control Protocol: KATCP}
\label{sec:KATCP}

\noindent
The MeerKAT telescope contains a large number of devices that require
control and monitoring, usually both. Most of these devices are
bespoke, meaning that they were constructed either within the project
or by a contractor to the specifications defined by the project.  As a
consequence we had the luxury of being able to select a communications
design strategy, architecture and protocol suite used for the control
and monitoring of the telescope.

A significant design decision was to confine low-latency, real-time
control functions to a small set of components. In other words the
control system as a whole does not operate on a low-latency basis ---
instead, time critical commands are sent out well in advance and are
scheduled by a smaller real-time component close to the limited number
of elements requiring such control. The system distributes a clock
signal to subordinate nodes.  This decision makes it possible to use
computer systems running conventional (not hard-real-time) operating
systems that communicate using common, non-deterministic interconnects
and protocols. In particular it was possible to select TCP/IP over
Ethernet for transport: these are well-established, pervasive and
inexpensive lower communications layers.

The only locally developed part of the control network is the
application control protocol, KATCP, running on top
of TCP/IP\@. KATCP borrows from established, classical Internet
protocols such as SMTP, FTP and POP in that is an application protocol
which is easily read by humans; each line of text is a protocol
message. The protocol is not too different from a
command prompt or shell interface, where each input line is also a
command. However, unlike command line utility output, the protocol
replies have a regular, well-defined structure\footnote{See
\url{https://casper.berkeley.edu/wiki/KATCP}
 and \url{https://pythonhosted.org/katcp}.
}.


\subsection{Streaming Protocol for Exchanging Astronomical Data: SPEAD}

\noindent
In analogy with the control protocol, we unified the data exchange
protocol wherever practicable. Hence the correlator internals,
correlator output stage, and science processor all share a common
data-exchange format called
SPEAD\footnote{The specification document and a full description can
be found at \url{https://casper.berkeley.edu/wiki/SPEAD}}. 


\textcolor{blue}{
SPEAD is a one-way, best-effort, self-describing protocol, containing
both machine and human-readable descriptions. This allows receiver
processes to automatically unpack data, enabling application designers
to concern themselves with algorithms rather than with data
exchange. It also allows users to comment their data-streams so that
stored data can easily be reinterpreted at a later date without
additional documentation, or to easily debug data exchanges.}

Other data protocols such as the VLBI Data Interchange Format, 
VDIF~\footnote{\url{http://www.vlbi.org/vdif/}} were considered, 
\textcolor{blue}{but it was decided
that they were not sufficiently self-describing or flexible.
}
SPEAD was
developed in collaboration with the PAPER (Precision Array for Probing
the Epoch of Reionization; \citealt{parsons}) team as part of the
broader CASPER collaboration and is available under an open-source
(GPL) license. SPEAD can be used for on-the-wire exchange, for on-disk
storage or for piping data between application processes within a
single compute node. 
This protocol is implemented on ROACH boards, Central Processing
units (CPUs) and Graphical Processing Unit (GPUs).

SPEAD is in essence optimised for the transmission of arrays of data
from one location to another. In particular we use SPEAD extensively
for the movement of Python Numpy arrays (Figure~\ref{fig:SPEAD}).  It
is designed to allow exchange of arbitrary data structures and aims to keep
receivers' copies fresh by propagating any changes
to the receiver as these changes occur on the transmitter.
\textcolor{blue}{
The metadata and variable descriptors are injected into the primary data stream.
Such changes will occur if the correlator is started, or changes mode
or the frequency of a local oscillator is altered.
Receiver processes keep state information, so
the latest values of all variables within the structure are always
available, even if only the changed subset is transmitted.
If these dynamic features are not required, a lightweight and faster
static implementation can be made.}

\label{sec:SPEAD}
\begin{figure}
\includegraphics[width=\columnwidth]{figs/f8.eps} 
\caption{SPEAD exchanges between a range of platforms. Once the data has been sampled and Fourier transformed in the F-engine all the data is handled internally as SPEAD packets or as Numpy arrays.}
\label{fig:SPEAD}
\end{figure}

As a one-way transport, and because it operates over UDP on Ethernet
networks, it is easy to support multicast transparently, allowing
multiple devices to subscribe to the same data-streams.
\textcolor{blue}{This allows
real-time data inspection and plotting by subscribing to a subset of the 
data. As UDP does not check for packet delivery SPEAD was made to tolerate
some packet loss, but cannot request a lost or corrupted packet.
Buffering helps by allowing for packets being received out of sequence.}


