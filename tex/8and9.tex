\section{First Image with KAT-7}
\label{sec:images}

\noindent
The first tests of the completed KAT-7 system were undertaken in 2011
August using the Wideband continuum mode (See
Table~\ref{table:modes}). The specific combination of sensitivity,
angular resolution, $\approx$ 3(4)$\arcmin$ at 1822(1328)\,MHz, and
field-of-view ($\approx 1\degr$) of KAT-7 require extended southern
sources for use as imaging targets to use for testing this mode.  One
such source is \mbox{PKS 1610-60} which has an integrated flux density of
43\,Jy at 1822\,MHz, and an extent of 30\arcmin\ in right ascension and
4\arcmin\ in declination \citep{christiansen}.  This source was
observed with KAT-7 for 10 hours with both polarizations at a
frequency of 1822\,MHz. The KAT-7 correlator outputs the data into an
observatory-specific format based on HDF5.  After conversion to a
standard Measurement Set the data are calibrated and imaged in CASA
\citep{mcmullin}. The final total-intensity image can be found in
Figure~\ref{fig:pks1610}. The dynamic range (ratio of the peak image
intensity to the r.m.s.~image noise) of this map is about 200:1.


\begin{figure}
\includegraphics[width=\columnwidth]{figs/f9.eps} 
\caption{Synthesis image of \mbox{PKS 1610-60} region containing the more
  prominent bent-double radio galaxy \mbox{PKS 1610-60.8} (below) and the
  smaller radio galaxy \mbox{PKS 1610-60.5} (above).}
\label{fig:pks1610}
\end{figure}



\section{Radio Continuum Imaging and Monitoring}
\label{sec:continuum}

\noindent
A sample of calibrators ($-90\degr<\delta<30\degr$ ) are continuously
being monitored and identified as possible short-spacings flux-density
calibrators \citep{kassaye}. The aim of this monitoring campaign is
also to image potential new calibrators with short spacings and to
assess their long-term variability. Several sources that are suitable
as point-source calibrators for interferometers with larger dishes,
and consequently smaller fields-of-view (such as e.g. \mbox{J0943-081}
and \mbox{J2326-4027}) have proven unsuitable for KAT-7 as they have
nearby contaminating sources within the primary beam.
%re-written -Tony
In general we have rejected candidate calibrators that are found to
have a contaminating source above 10\% of their peak
\textcolor{blue}{brightness}, or where the integrated flux density
(estimated using the shortest baseline) is more than 10\% greater than
the maximum \textcolor{blue}{flux density per beam}
%
% Most placed on the sky will have a few hundred milliJansky spread
% over a few sources within the KAT-7 primary beam at 1.2-1.95\,GHz.
\textcolor{blue}{Given the size of the KAT-7 primary beam at
  1.2-1.95\,GHz, most maps have few hundred milliJansky spread over a
  number of sources brighter than 1mJy/beam. This is consistent source
  number counts from the FIRST and ATLBS surveys.}  With the modest
angular resolution of KAT-7 we would expect our images to be limited
by confusion noise rather than thermal noise in Stokes
\emph{I}. \textcolor{blue}{For most places on the sky} we hit a
continuum confusion limit of about 1(2) mJy/beam at 1822(1328)\,MHz
(based on no more than one source per five beams), and making high
dynamic range continuum images where confusion is so high has proven
difficult. \textcolor{blue}{The confusion limit is even higher along
  the galactic plane.}

\textcolor{blue}{ From a full-polarization analysis, the polarization
  fraction of $8.4\pm0.6 \%$ was seen for 3C286 and $8.1\pm0.6 \%$ for
  3C138. These values are in good agreement with those determined by
  \citet{perley}.  Polarization was calibrated using full-Stokes
  observations of unpolarized radio sources such as \mbox{PKS
    1934-638} to determine instrumental leakage (of total intensity
  into polarization), and sources with high linear polarization at
  L-band over a large range of parallactic angles to measure delay and
  phase offsets between horizontal and vertical dipoles.  Typical
  leakages within the half-power beam width were of the order of 3\%
  in the upper band (1822\,MHz) and 3--6\% in the lower band
  (1322\,MHz), so second order correction effects were negligible.}

\textcolor{blue}{ Although the dynamic range in continuum total
  intensity is limited by confusion after about 8 hours, this is not
  the case for polarization or spectral line observations. We do
  however need the instrumental polarization corrections for high
  dynamic range spectral line observations, such as in OH maser
  observations.}


\section{Commissioning Science with KAT-7}
\label{sec:comm}

\noindent
KAT-7 went through in a science commissioning phase.  The requirements
for commissioning science projects are (i) that they test the
performance of various aspects of commissioning (e.g.~correlator
modes, data reduction pipeline, RFI mitigation), and (ii) they
endeavour to contribute new and original scientific results.  These
projects touch on a broad range of scientific interest, including, but
not limited to, topics identified as MeerKAT Key Science Projects.
Projects include:

\begin{itemize}
\item Monitoring variable continuum sources (selected from ATEL
  announcements).
\item Imaging extended, diffuse continuum emission.
\item Timing strong southern pulsars.
\item Conducting very long baseline interferometry (VLBI).
\item Measuring HI spectral line kinematics and low surface brightness emission in nearby galaxies and clusters.
\item Monitoring OH maser emission.
\item Mapping large scale polarization.
\item Implementing techniques for HI intensity mapping.
\end{itemize}

\noindent
We outline several of these completed and ongoing science
commissioning projects in the following subsections.  We highlight
their scientific importance, and their contributions to KAT-7
commissioning and preparations for MeerKAT.

\subsection{Transient sources}

\noindent

Many transient sources 
%(e.g.~PKS~1424-418, Circinus~X-1, PKS~2356-502)
have been monitored with KAT-7 ranging from neutron-star 
and black-hole X-ray binaries, blazars, galactic 
gamma-ray binaries, and novae.
Long-term monitoring of Circinus X-1, a bright and highly variable
X-ray binary, was carried out using KAT- 7 at 1.9\,GHz and the HartRAO
radio telescope at 4.8 and 8.5\,GHz by \citealt{armstrong}, and is
the first published scientific paper using KAT-7 data, and a good
example of multi-wavelength collaboration between South African
instruments. The observations confirm a return to strong radio flaring
first observed in 1970's that had been suppressed for two
decades. Figure~\ref{fig:2F_OBS} shows this long-term behaviour.

\begin{figure}
\includegraphics[width=\columnwidth]{figs/f10.eps}
\caption{A radio light curve of Circinus~X-1 obtained with KAT-7 (1.9\,GHz,
red crosses) and HartRAO (4.8\,GHz, black circles and 8.5\,GHz, blue stars). 
The X-ray binary system is clearly flaring to Jansky levels again. The times in 
which KAT-7 is operational are shown in green. The vertical bars at each
point represent the measurement noise, while the horizontal bars represent
the observation length.}
\label{fig:2F_OBS}
\end{figure}

Simultaneous observations of the black hole candidate Swift
\mbox{J1745-26} with KAT-7, the Very Large Array (VLA), and the Australia
Telescope Compact Array (ATCA) radio interferometers reveal a `failed outburst'
event, illuminating details of the complex processes of accretion and
feedback in black holes \citep{curran}.

The blazar \mbox{PKS 1424-418} was observed as follow-up to ATEL4770
(Figure~\ref{fig:atel4770}).  Similar observations have been carried
out using the blazar source \mbox{PKS 1510-089} \citep{oozeer},
whereby at least one flaring has been successfully recovered. Another
weak transient source, \mbox{PKS 2356-502}, was observed in 2012 May
and continues to be monitored (Figure~\ref{fig:2326}).

\begin{figure}
\includegraphics[width=\columnwidth]{figs/f11.eps}
\caption{\mbox{PKS 1424-418} imaged after ATEL4770. Contours start at $\pm$
  10\,mJy/beam and go in steps of a factor of 4. Negative contours are
  shown dashed and the synthesized beam is shown in the top left
  corner. No primary beam correction has been applied.}
\label{fig:atel4770}
\end{figure}

\begin{figure}
\includegraphics[width=\columnwidth]{figs/f12.eps}
\caption{\mbox{PKS 2356-502}. Contours start at $\pm$ 2.5\,mJy/beam and go in
  steps of a factor of 4. Negative contours are shown dashed and the
  synthesized beam is shown in the top left corner. 
  The FWHM of the primary beam is $1.0\deg$. No primary beam
  correction has been applied. 
}
\label{fig:2326}
\end{figure}




\subsection{Diffuse radio emission}

\noindent
With the availability of short spacings (26--185\,m, see
Table~\ref{table:array}), KAT-7 is an ideal instrument to detect
extended diffuse radio emission especially in galaxy clusters hosting
halos and relics. For example, Abell~3667 was observed with KAT-7 and
the nature of the diffuse emission (two large-scale relics) was confirmed
(see \citealt{riseley}). 
Figure~\ref{fig:aco} shows the continuum image of
Abell~3667 at 1822\,MHz.




\begin{figure}
\includegraphics[width=\columnwidth]{figs/f13.eps}
\caption{Total intensity map of Abell 3667 at 1826\,MHz. Image noise
  is 1.30\,mJy/beam. Contours mark [-1,1,2,3,4,6,8,12,16,32,64,128]
  $\mathrm{\times \sigma_{rms}}$. The synthesised beam size (empty circle bottom left) is
  $\mathrm{184 \times 150 \arcsec}$.
  The FWHM of the primary beam is $\mathrm{1.0\deg}$.
  }
\label{fig:aco}
\end{figure}


figure \subsection{Pulsars}

\noindent


KAT-7 high-time resolution capability was first demonstrated through a
single-dish observation of the Vela pulsar (\mbox{J0835-4510}) in late
2012. Here Nyquist-sampled raw-voltage data was written to disk and
processed offline using a custom pipeline which incorporates
{\small\textsc{dspsr}}\footnote{\url{http://dspsr.sourceforge.net/}}
and
{\small\textsc{psrchive}}\footnote{\url{http://psrchive.sourceforge.net/}}
tools \citep{hotan}. Semi-regular, dual-polarization observations of
bright pulsars ($\gtrsim 10$~mJy) with the KAT-7 beamformer have since
become standard practice. Single-pulse and folded searches have been
performed successfully on a number of sources. A series of short
pulsar timing campaigns have also been undertaken, predominantly
focusing on \mbox{J0835-4510} due to its low period ($\sim$89.3~ms)
and high flux density (1.1~Jy at 1400\,MHz;
\citealt{bf74}). Figure~\ref{fig:pulsarspec} shows a comparison
between the timing residuals obtained in 2014 November for
\mbox{J0835-4510} using HartRAO and KAT-7 data.  \textcolor{blue}{
  Figure~\ref{fig:pulsarfold} shows an observation of bright pulsar,
  \mbox{J0835-4510}. It was reduced and plotted \textsc{psrchive}
  software.}

\begin{figure}
\includegraphics[width=\columnwidth]{figs/f14.eps} 
\caption{Phase residuals for Vela (\mbox{J0835-4510}), as produced by
  the pulsar timing package \emph{tempo2} \citep{hobbs}. Those from
  the 26~m HartRAO telescope (observed almost daily) are plotted in
  red, whilst the blue points show observations made with the KAT-7
  beamformer.  The plot confirms that the timing is consistent between
  the two radio telescopes.}
\label{fig:pulsarspec}
\end{figure}

\begin{figure}
\includegraphics[width=\columnwidth]{figs/f15.eps} 
\caption{Observation of the bright pulsar, \mbox{J0835-4510}, done
  with KAT-7. These were taken for several minutes and this plot
  presents the results after folding at the pulsar period of
  89~milliseconds. The horizontal axis show the pulse phase and the
  lower figure shows power as a function of frequency, with integrated
  power over the band plotted above, while the upper figure shows
  pulse power as function of time. The darker colour denotes higher
  power.}
\label{fig:pulsarfold}
\end{figure}


\subsection{VLBI}

\noindent
Pathfinder VLBI observations are critical to the establishment of the
African VLBI Network (AVN; \citealt{gaylard}) in advance of SKA1.
Although KAT-7 is not equipped with a hydrogen maser, typically
required for VLBI observations, we obtained fringes between a single
KAT-7 antenna and the 26-metre antenna located at HartRAO near
Johannesburg, South Africa. The local GPS-disciplined rubidium time
source was deemed sufficient for this experiment.  A short 30-minute
test observation of the bright source 3C273 was performed on 2010
November 23. This involved HartRAO recording the data to their Mark~V
VLBI system, and KAT-7 recording to disk using the raw-voltage capture
system.

Due to the sampling clock of KAT-7 not being a power of 2, we used a
DDC (digital down converter) to convert the sample rate in order to
produce the resultant 64\,MHz bandwidth, via CUDA code deployed to a
GPU.
 
The data were combined and reduced using in-house python scripts. The
data were then verified using the DiFX correlator package
\citep{Deller}, with strong fringes subsequently detected
(Figure~\ref{fig:vlbifringe}).


\begin{figure}
\includegraphics[width=\columnwidth]{figs/f16.eps} 
\caption{VLBI fringes obtained between KAT-7 and HartRAO using DiFX
  software correlator. The horizontal axis is frequency channel and
  the plots in descending order are amplitude, phase (in degrees) and
  lag.}
\label{fig:vlbifringe}
\end{figure}



\subsection{Spectral Lines: HI and OH observations}

\subsubsection{HI}

\noindent
The receiver on KAT-7 covers the 1420\,MHz transition of neutral
hydrogen (HI) redshifted as far as 1200\,MHz (z=0.184),
although the intrinsic faintness of the
spectral line and the resolution of the telescope make it most
suitable to emission-line studies in the nearby Universe. In
particular, KAT-7 is well suited for the study of nearby southern
objects which have emission on scales larger than 15$\arcmin$,
typically invisible to telescopes such as the VLA that lack the short
baselines. This is well illustrated in the KAT-7 HI observations of
NGC~3109 \citep{Carignan}, where 40\% more HI flux was detected
than in previous VLA measurements (see Figure~\ref{fig:ngc3109}). The
new data also allowed the derivation of the rotation curve 16$\arcmin$
further out than the previous VLA measurements (see Figure 10 of
\citealt{Carignan}). These observations were undertaken as part of
commissioning the 6.25\,MHz spectral-line correlator mode, `HI 
Galaxies' (Table~\ref{table:modes}). They led to the
discovery of a source of antenna-dependent, faint, very narrow
internally-generated RFI that was successfully eliminated by the
insertion of a low-pass filter along the signal path.

The wide field-of-view of KAT-7 and wide-bandwidth spectral-line modes
make it a powerful instrument for mosaicking large areas on the sky,
competitive with ATCA in terms of total observing time for similar
sensitivity.  As part of commissioning the 25\,MHz `HI Galaxy
Clusters' spectral-line correlator mode, several projects
to image the HI in galaxy groups and clusters are currently underway
(e.g.~Antlia Cluster, NGC~4055 Group). In particular, the Antlia
Cluster was observed for 147 hours, in a 7-pointing discrete hexagonal
pattern, covering roughly $4.4\mathrm{deg^2}$ , and reaching 
$0.97 \mathrm{mJy \,beam^{-1}}$
over the $15.5 \mathrm{km \,s^{-1}}$ channels, see Figure~\ref{fig:antlia2} \citealt{hess}.  This deep map detected 30 HI cluster members, 28 of
which were new HI detections, and 20 of which were the first redshift
measurements at any wavelength.  The broad bandwidth coverage, and
lack of bright, dominating sources in the Antlia mosaic led to the
recognition of the `$u=0$' problem in KAT-7 spectral-line data.  In
this case, RFI that exists below the level of the noise in the
visibility data, and would normally be incoherent, adds constructively
when the fringe rate between two antennas is equal to zero.  It is not
a unique phenomenon to KAT-7, but it is particularly a problem on
short baselines.  (See \citealt{hess} for a more complete explanation).

\begin{figure}
\includegraphics[width=\columnwidth]{figs/f17.eps}
\caption{HI mosaic of NGC~3109 and Antlia dwarf superposed on a
 DSS~B image. The green contours are [0.5, 1.07, 2.1, 3.2, 5.4,
 10.7, 21.4, 32.2, 53.6, 75.0, and 96.5] times the peak of 
 $107.2 \mathrm{Jy \,{beam}^{-1} \times km \,s^{-1}}$.
 NGC~3109 ($\mathrm{V_{sys}  = 404 \,km\, s^{-1}}$) is at the
 top and Antlia ($\mathrm{V_{sys} = 360 \,km\,s^{-1}}$) is at the bottom. The two
 background galaxies ESO~499-G037 ($\mathrm{V_{sys}  = 953\, km \,s^{-1}}$) and
 ESO~499-G038 ($\mathrm{V_{sys} = 871\, km \, s^{-1}}$) with its associated HI cloud
 ($\mathrm{V_{sys} = 912 \,km \,s^{-1}}$) are between NGC~3109 and Antlia. The
 synthesized beam is shown in the lower-right corner.}
\label{fig:ngc3109}
\end{figure}

\begin{figure}
\includegraphics[width=\columnwidth]{figs/f18.eps}
\caption{A cut-out of the Antlia Cluster mosaic: color contours
  corresponding to the systemic velocity of the galaxy detected in HI
  emission are overlaid on a WISE 3.4\,$\mu$m image. NGC~3281,
  detected in absorption, is in dashed contours.  Dark blue is
  $\mathrm{<2200 \,km \, s^{-1}}$, cyan is $\mathrm{2200-2800 km
    \,s^{-1}}$, magenta is $\mathrm{2800-3400 \,km \,s^{-1}}$, and red
  is $\mathrm{>3400 \,km\, s^{-1}}$.  HI rich galaxies are detected in
  a ring around the dominant elliptical galaxy, NGC~3268. These KAT-7
  observations had a usable bandwidth sensitivity to HI emission at
  recessional velocities between $\mathrm{1200-4600 \,km \,s^{-1}}$,
  and 30 HI galaxies were detected between $\mathrm{1800-4300 \,km
    \,s^{-1}}$.  For more details see \citealt{hess}}
\label{fig:antlia2}
\end{figure}

\subsubsection{OH}

\noindent 
The frequency range of KAT-7 covers all four ground-state transitions
of hydroxyl: 1612, 1665, 1667 and 1720\,MHz.  Maser emission occurs in
these transitions under a variety of conditions: in massive
star-forming regions (primarily in the main-lines at 1665 and
1667\,MHz \citep{caswell98}, in the shells around AGB stars at
1612\,MHz \citep{sevenster}, and in shocked regions such as supernova
remnants at 1720\,MHz \citep{wardle}.

While KAT-7 does not have the angular resolution to map individual
maser spots, its narrowest spectral mode gives a velocity resolution
of 68\,m/s, enough to resolve narrow maser lines in massive
star-forming regions.  A number of methanol masers in star-forming
regions have been found to exhibit periodic variations
(\citealt{goedhart} and references therein). \citet{green} attempted
to monitor a short-period source in the hydroxyl main-lines and found
a weak indication of periodicity, but the time-series was
undersampled. Monitoring is an ideal niche application for KAT-7 since
it generally has more time available than the Parkes Telescope or
ATCA\@. Six of the known periodic methanol masers have hydroxyl maser
counterparts that can be detected with KAT-7. These sources are
being monitored on a weekly basis at both 1665 and 1667\,MHz, using
interleaved observations at both frequencies in a 13-hour schedule
block.

\begin{figure*}
\includegraphics[width=\textwidth]{figs/f19.eps}
\caption{Masers in \mbox{G331.13-0.24} at 1665\,MHz monitored with KAT-7. The
  underlying greyscale map shows the positions of the continuum
  emission while the coloured contours show the continuum-subtracted
  maser emission. The insets show the spectrum at the indicated
  positions, where the radial velocity scale is $\mathrm{m \, s^{-1}}$ with
  respect to
  the Local Standard of Rest. The synthesized beam is shown as a
  dashed ellipse at the bottom left corner.}
\label{fig:masers}
\end{figure*}

Figure~\ref{fig:masers} shows a pilot observation of the star-forming
region \mbox{G331.13-0.24} at 1665\,MHz. The large field-of-view of KAT-7 led
to the detection of several maser sources as well as several diffuse
HII regions. The positions, velocities and flux densities are
consistent with those from recent Parkes observations
\citep{caswell14}, confirming that the system is performing as
expected.  The observations are dynamic-range-limited and three of the
six fields observed (including \mbox{G331.13-0.24}) are dominated by off-axis
sources, potentially creating errors in the measured flux density of
the target source.  Direction-dependent calibration techniques are
still under investigation \citep{bhat}. Preliminary results from the
fields not affected by these effects show significant variability
(Goedhart et al.~in~prep). Observations of \mbox{G330.89-0.36}, which has a
peak brightness of 824\,Jy\,per beam, were used to measure the spectral
dynamic range and channel isolation. The line-to-line dynamic range
was \textcolor{blue}{31\,dB. A maximum imaging dynamic range of 32\,dB} was
achieved after applying self-calibration and \textsc{clean}ing
interactively.



%=================================================
\iffalse 
\subsection{HI Intensity Mapping}
\noindent
In cosmology, intensity mapping of the HI spectral line is an
observation technique used for surveying the large-scale structure of
the universe (see e.g.~\citealt{bull}). By mapping the distributions
of neutral hydrogen gas in nearby galaxies astronomers are able to
model the dark matter distribution of each galaxy.

KAT-7 records both single-dish (auto-correlation) data as well as
interferometric (cross-correlation) products. Auto-correlation data
can be used for creating the large-scale high-resolution maps, while
the interferometric data allows for the removal of systematic errors
through calibration.

This implementation of combined single dish and interferometric data
has been identified as a possible observation mode for improved
detection of the Baryon Acoustic Oscillation (BAO) feature and KAT-7
may be used as a proof of concept for
future HI intensity mapping surveys (see \citealt{bull}). 
\fi
%============================

\section{Technology Lessons Learned}
\label{sec:lessons}

\noindent
We learned several lessons during the construction and use of the
KAT-7 array.

The lightweight Stirling coolers used for the cryogenic system are
cheaper than conventional Gifford-McMahon (G-M) cycle coolers but need
maintenance (at least) annually. They are not cold enough (80\,K) to
achieve a large cryopumping effect and so the system needs ion pumps
in order to retain a high vacuum and mechanical dampers to reduce the
vibration generated by the Stirling coolers. Both of these sub-systems
need replacement on a regular basis, which involves the use of cranes
and a team of technicians, plus about 2 days on an external pump to
obtain high enough vacuum before the receivers can be used.  The
capital expenditure for Stirling cycle cooling for 3 receivers on all
64 MeerKAT antennas is ZAR 3.3 million less, but the cost of expected
maintenance, parts and electricity bill would be ZAR 1.1 million
more. This means that over a 3-4 year period Stirling cycle coolers
would end up much more expensive to install and operate than G-M
cryogenic coolers. Using the G-M cooling system should also reduce
down-time for MeerKAT antennas and this has not been factored into the
costs.

The composite dishes with embedded metal mesh have large weight
advantages over dishes made of large metal panels which in turn means
that the backing structure can be light and the motors need not be
powerful.  They must be constructed correctly as they cannot be
adjusted or machined after the fact, and this is in turn means that
the fibre weave, temperature, humidity and vacuum need to be very
carefully controlled during fabrication.  Also, the mould needs to be
machined to higher accuracy than is needed for the dish; this is
relatively simple for a circularly symmetrical design, as only one
segment has to be accurately made and then copied.  This is far more
difficult for a design without that symmetry (e.g.~MeerKAT).  Large
single-piece dishes must be made on-site as moving them becomes a
logistical problem.


The ball-screw mechanism for the elevation drive has proven reliable
and simple and it is retained in the MeerKAT design.

The KATCP protocol has shown itself to be flexible and very useful in
testing and commissioning, and can readily be put into scripting form
for routine observations.

SPEAD, although still undergoing active development, has also proven
effective and will be further rolled out for use in MeerKAT.

The correlator architecture, correlator and beamformer based on FPGA
technology and data transfer over a managed network switch, have also
been found to be both flexible and reliable.

For continuum sources we reach a confusion limit of about
0.5--1.0\,mJy (depending on the field) after 8\,hours; and this is
worse in the Galactic plane.  This makes high-dynamic-range imaging
difficult, although dynamic ranges of a few 1000 are regularly
reached. For spectral-line observations this confusion limit is not an
issue.

In summary, KAT-7 is well suited for:

\begin{itemize}
\item Mapping low-surface-brightness continuum emission (e.g.~radio
  halos and relics).
\item Mapping extended ($\geq 10\arcmin$) spectral-line sources, such
  as HI around nearby galaxies and OH emission around star-forming
  regions. The maximum baseline is too short to measure individual
  maser spots, but we can measure their spectra with high frequency
  resolution.
\item Observing the variability of continuum sources $\geq 10$\,mJy at
  1.8\,GHz.
\item Pulsar observations with the beamformer output.
\item VLBI observations with the beamformer output.
\end{itemize}

