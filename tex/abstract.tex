\section{Introduction}
\noindent

KAT-7 comprises seven antennas
\footnote{\url{https://sites.google.com/a/ska.ac.za/public/kat-7}},
each with a diameter of 12\,m, and is located near the SKA core site
in the Karoo semi-desert area, about 80\,km north-west of Carnarvon in
the Northern Cape.  \textcolor{blue}{Figure~\ref{fig:aerial} shows a
  aerial photograph of the array and} Figure~\ref{fig:layout} shows
the plan of the KAT-7 array to give the scale. Baselines range in
length from 26\,m to 185\,m. The key parameters of KAT-7 are listed in
Table~\ref{table:array} and the correlator modes in
Table~\ref{table:modes}.

\begin{figure}
\includegraphics[width=\columnwidth]{figs/f1.eps}
\caption{An aerial view of the KAT-7 array; the antennas are pointing 
towards the south.}
\label{fig:aerial}
\end{figure}


\begin{figure} 
  \includegraphics[width=0.99\columnwidth]{figs/f2.eps}
  \caption{KAT-7, the 7-dish Karoo Array Telescope layout; each block
    is 10~m and the dish sizes are to scale. The zero point (where the
    thicker lines cross) is the array centre at $30.7148\degr$ S,
    $21.388\degr$ E.}
\label{fig:layout}
\end{figure}

Originally the Karoo Array Telescope (KAT) was planned to consist of
20 antennas, and the main objective of building a radio telescope was
to support South Africa's bid to host the international Square
Kilometre Array (SKA). The scope of the project was later expanded to
build the 64-antenna MeerKAT array.  Before building MeerKAT it was
decided to build a smaller prototype array to field-test some of the
technologies that might be used in MeerKAT; that array is KAT-7.

In addition to hosting the SKA and building MeerKAT, a further focus
area of the SKA South Africa project was to embark on a wide-ranging
Human Capital Development program, to ensure that a new generation of
scientists and engineers would be available to use the MeerKAT and SKA
radio telescopes, and to further science and engineering in South
Africa in general.


The acquisition process of KAT-7 began in 2008 with the writing of the
telescope requirements specification. `First light' fringes, which
were the first sucessful observations by the interferometer, were obtained
between two antennas in 2009. This effectively marked the beginning of
the commissioning process. \textcolor{blue}{The commissioning and user
  verification process is complete, and KAT-7 is currently operational.}

This paper is laid out as follows. In sections~\ref{sec:design}
and~\ref{sec:freq} we discuss the design drivers for the project. We
then detail the antennas and optics (section~\ref{sec:ant}), the Radio
Frquency and Intermediate Frequency (RF/IF) chain
(section~\ref{sec:RF}), the correlator (section~\ref{sec:corr}) and
control systems (section~\ref{sec:control}). In
sections~\ref{sec:images}--\ref{sec:comm} we describe some of the
early science and commissioning observations. We summarize lessons
learned in section~\ref{sec:lessons}.


