\documentclass[usenatbib,usegraphicx]{mn2e}
\usepackage{amsmath}
\usepackage{amssymb}
\usepackage{color}
\usepackage{graphicx}
\usepackage{graphics}
\usepackage{natbib}
\usepackage{enumerate}
\usepackage{hyperref}
\usepackage{times}
%       \citestyle{aa}
\usepackage{float}

\hyphenation{HartRAO}
\def\aap{Astronomy \& Astrophysics}
\def\ApJ{The Astrophysical Journal}
\def\AJ{The Astronomical Journal}
\def\MNRAS{Monthly Notices of the Royal Astronomical Society}

\begin{document}

\bibliographystyle{mn2e}

\title[The KAT-7 Radio Telescope] {Engineering and Science Highlights of the KAT-7 Radio Telescope}

\date{Accepted. Received ; in original form }

\pagerange{\pageref{firstpage}--\pageref{lastpage}} \pubyear{2015}
\label{firstpage}

\author[A.~R.~Foley et al.]
{A.~R.~Foley\thanks{E-mail:tony@ska.ac.za},$^1$ 
T.~Alberts,$^1$ 
R~P.~Armstrong,$^{1,2}$ 
A.~Barta,$^1$ 
E.~F.~Bauermeister,$^1$
H.~Bester,$^1$ 
\newauthor
S.~Blose,$^1$ 
R.~S.~Booth,$^1$
D.~H.~Botha,$^3$
C.~Carignan$^4$
T.~Cheetham,$^1$
K.~Cloete,$^1$
G.~Coreejes,$^1$
\newauthor
R.~C.~Crida,$^1$
S.~D.~Cross,$^1$
F.~Curtolo,$^1$
A.~Dikgale,$^1$ 
M.~S.~de~Villiers,$^1$
L.~J.~du Toit,$^3$
\newauthor
S.~W.~P.~Esterhuyse,$^1$
B.~Fanaroff,$^1$
R.~P.~Fender$^{2,4}$
M.~Fijalkowski,$^1$
D.~Fourie,$^1$
B.~Frank,$^{1,5}$
\newauthor
D.~George,$^1$
P.~Gibbs,$^1$
S.~Goedhart,$^1$
J.~Grobbelaar,$^1$
S.~C.~Gumede,$^1$
P.~Herselman,$^1$
\newauthor
K.~M.~Hess,$^{4,5,6}$
N.~Hoek,$^1$
J.~Horrell,$^1$
J.~L.~Jonas,$^{1,7}$
J.~D.~B.~Jordaan,$^3$
R.~Julie,$^1$
F.~Kapp,$^1$
\newauthor
P.~Kotz\'e,$^1$
T.~Kusel,$^1$
A.~Langman,$^{1,8}$
R.~Lehmensiek,$^3$
D.~Liebenberg,$^1$
I.~J.~V.~Liebenberg,$^3$
\newauthor
A.~Loots,$^1$
R.~T.~Lord,$^1$
D.~M.~Lucero,$^4$
J.~Ludick,$^1$
P.~Macfarlane,$^1$
M.~Madlavana,$^1$
\newauthor
L.~Magnus,$^1$
C.~Magozore,$^1$ 
J.~A.~Malan,$^1$
J.~R.~Manley,$^1$
L.~Marais,$^1$
N.~Marais,$^1$
S.~J.~Marais,$^3$
\newauthor
M.~Maree,$^1$
A.~Martens,$^1$
O.~Mokone,$^1$
V.~Moss,$^1$
S.~Mthembu,$^1$ 
W.~New,$^1$
G.~D.~Nicholson,$^9$ 
\newauthor
P.~C.~van~Niekerk,$^3$ 
N.~Oozeer,$^1$
S.~S.~Passmoor,$^1$
\textcolor{blue}{A.~Peens-Hough,$^1$}
A.~B.~Pi\'nska,$^1$
P.~Prozesky,$^1$ 
\newauthor
S.~Rajan,$^1$
S.~Ratcliffe,$^1$ 
R.~Renil,$^1$
L.~L.~Richter,$^1$
D.~Rosekrans,$^1$
A.~Rust,$^1$
A.~C.~Schr\"oder,$^{10}$
\newauthor
L.~C.~Schwardt,$^1$
S.~Seranyane,$^1$
\textcolor{blue}{M.~Serylak,$^{14,15}$}
D.~Shepherd,$^{11}$
R.~Siebrits,$^1$
L.~Sofeya,$^1$ 
\newauthor
R.~Spann,$^1$
R.~Springbok,$^1$
P.~Swart,$^1$
Venkatasubramani~L.~Thondikulam,$^1$
I.~P.~Theron,$^3$ 
\newauthor
A.~Tiplady,$^1$
O.~Toruvanda,$^1$
S.~Tshongweni,$^1$ 
L.~van~den~Heever,$^1$
C.~van~der~Merwe,$^1$
\newauthor
R.~van~Rooyen,$^1$
S.~Wakhaba,$^1$
A.~L.~Walker,$^1$
M.~Welz,$^1$
L.~Williams,$^1$ 
M.~Wolleben,$^{1,12}$
\newauthor
P.~A.~Woudt,$^4$
N.~J.~Young,$^{1,13}$  
J.~T.~L.~Zwart $^{4,14}$
\\
$^1$Square Kilometer Array South Africa, The Park, Park Road, Pinelands, Cape Town 7405, South Africa \\
\textcolor{blue}{$^2$Deptartment of Physics, University of Oxford, Oxford, UK} \hfill
$^3$EMSS Antennas, 18 Techno Avenue, Technopark, Stellenbosch 7600, South Africa\\
$^4$Astrophysics, Cosmology and Gravity Centre, Dept. of Astronomy, University of Cape Town, Cape Town, South Africa\\
$^5$Netherlands Institute for Radio Astronomy (ASTRON), PO Box 2, 7990 AA Dwingeloo, The Netherlands\\
$^6$Kapteyn Astronomical Institute, University of Groningen, PO Box 800, 9700 AV Groningen, The Netherlands\\
$^7$Rhodes University, Grahamstown, South Africa \hfill
$^8$Dept. of Electrical Engineering, University of Cape Town, Cape Town, South Africa\\
$^9$Hartebeesthoek Radio Astronomy Observatory South Africa\hfill
$^{10}$South African Astronomical Observatory, P.O. Box 9, Observatory 7935,South Africa\\
$^{11}$National Radio Astronomy Observatory,Socorro, New Mexico, U.S.A \hspace{0.8cm}
$^{12}$Dept. of Physics and Astronomy, University of Calgary, Alberta, Canada\\
$^{13}$School of Physics, University of the Witwatersrand, PO BOX Wits, Johannesburg 2050, South Africa\\
\textcolor{blue}{$^{14}$Department of Physics \& Astronomy, University of the Western Cape, Cape Town, South Africa}\\
\textcolor{blue}{$^{15}$Station de Radioastronomie de Nan\c{c}ay, Observatoire de Paris, PSL Research University, CNRS, Univ. Orl\'{e}ans, OSUC, 18330 Nan\c{c}ay, France}
}

\maketitle

\begin{abstract}
The construction of the KAT-7 array in the Karoo region of the
Northern Cape in South Africa was intended primarily as an
engineering prototype for technologies and techniques applicable to
the MeerKAT telescope. This paper looks at  the main
engineering and scientific highlights from this effort, and
discusses their applicability to both MeerKAT and other
next-generation radio telescopes. 
\textcolor{blue}{In particular we found that the
composite dish surface works well, but it becomes complicated to
fabricate for a dish lacking circular symmetry; the Stirling cycle
cryogenic system with ion pump to achieve vacuum works but demands
much higher maintenance than an equivalent Gifford-McMahon cycle
system; the ROACH (Reconfigurable Open Architecture Computing 
Hardware)-based correlator with SPEAD (Streaming Protocol for 
Exchanging Astronomical Data) protocol data transfer
works very well and KATCP (Karoo Array Telescope Control Protocol)
control protocol has proven very flexible and 
convenient.
KAT-7 has been used for scientific observations where it
has a niche in mapping
low surface-brightness continuum sources, some extended HI
 halos and OH masers 
around extended star-forming regions. It can also be used to monitor
continuum source variability, observe pulsars and make VLBI observations.}
\end{abstract}

\begin{keywords}
instrumentation:interferometers --  radio continuum:general -- radio lines:general
\end{keywords}
%\section{Introduction}
\section{Introduction}
\noindent

KAT-7 comprises seven antennas
\footnote{\url{https://sites.google.com/a/ska.ac.za/public/kat-7}},
each with a diameter of 12\,m, and is located near the SKA core site
in the Karoo semi-desert area, about 80\,km north-west of Carnarvon in
the Northern Cape.  \textcolor{blue}{Figure~\ref{fig:aerial} shows a
  aerial photograph of the array and} figure~\ref{fig:layout} shows
the plan of the KAT-7 array to give the scale. Baselines range in
length from 26\,m to 185\,m. The key parameters of KAT-7 are listed in
Table~\ref{table:array} and the correlator modes in
Table~\ref{table:modes}.

\begin{figure}
\includegraphics[width=\columnwidth]{figs/f1.eps}
\caption{An aerial view of the KAT-7 array; the antennas are pointing 
towards the south.}
\label{fig:aerial}
\end{figure}


\begin{figure} 
  \includegraphics[width=0.99\columnwidth]{figs/f2.eps}
  \caption{KAT-7, the 7-dish Karoo Array Telescope layout; each block
    is 10~m and the dish sizes are to scale. The zero point (where the
    thicker lines cross) is the array centre at $30.7148\degr$ S,
    $21.388\degr$ E.}
\label{fig:layout}
\end{figure}

Originally the Karoo Array Telescope (KAT) was planned to consist of
20 antennas, and the main objective of building a radio telescope was
to support South Africa's bid to host the international Square
Kilometre Array (SKA). The scope of the project was later expanded to
build the 64-antenna MeerKAT array.  Before building MeerKAT it was
decided to build a smaller prototype array to field-test some of the
technologies that might be used in MeerKAT; that array is KAT-7.

In addition to hosting the SKA and building MeerKAT, a further focus
area of the SKA South Africa project was to embark on a wide-ranging
Human Capital Development program, to ensure that a new generation of
scientists and engineers would be available to use the MeerKAT and SKA
radio telescopes, and to further science and engineering in South
Africa in general.


The acquisition process of KAT-7 began in 2008 with the writing of the
telescope requirements specification. `First light' fringes, which
were the first sucessful observations by the interferometer, were obtained
between two antennas in 2009. This effectively marked the beginning of
the commissioning process. \textcolor{blue}{The commissioning and user
  verification process is complete, and KAT-7 is currently being
  used.}

This paper is laid out as follows. In sections~\ref{sec:design}
and~\ref{sec:freq} we discuss the design drivers for the project. We
then detail the antennas and optics (section~\ref{sec:ant}), the Radio
Frquency and Intermediate Frequency (RF/IF) chain
(section~\ref{sec:RF}), the correlator (section~\ref{sec:corr}) and
control systems (section~\ref{sec:control}). In
sections~\ref{sec:images}--\ref{sec:comm} we describe some of the
early science and commissioning observations. We summarize lessons
learned in section~\ref{sec:lessons}.





\begin{table*}
\caption{Key performance parameters}
\begin{tabular}{lc}
\textbf{Parameters} & \textbf{Value} \\
\hline
Number of antennas        & 7 \\
Dish diameter             & 12\,m \\
Baselines                 & 26\,m to 185\,m \\
Frequency Range           & 1200\,MHz--1950\,MHz \\
Instantaneous Bandwidth   & 256\,MHz \\
Polarization              & Linear non-rotating (Horizontal + Vertical) feed \\
$T_{sys}$                 & \textless 35\,K across the entire frequency band \\
                         & ($\approx 30$\,K for all elevation angles \textgreater $30\degr$)\\
Antenna efficiency at L-band & 66\% \\
Primary beam FWHM at 1.8\,GHz & $1.0\degr$ \footnotemark[1] \\
Angular resolution at 1.8 GHz & 3\,arcmin \\
Location                 & latitude $30.7148\degr$ S, longitude $21.388\degr$ E, Elevation 1054m \\
Continuum Sensitivity    & 1.5\,mJy in 1 minute (256\,MHz bandwidth, $1\sigma$)\\
Angular Scales           & 3$\arcmin$ to 22$\arcmin$ \\
%\enddata
\hline
\end{tabular}
\\
$^1$The primary beam FWHM $\theta = 1.27\frac{\lambda}{D}$.\\
\label{table:array}
\medskip
\end{table*}
%table 2


\begin{table*}
\caption{Correlator Modes}
\begin{tabular}{lcc}
\textbf{Mode} & \textbf{Processed bandwidth/MHz} & \textbf{Channel bandwidth/kHz}  \\
\hline
Wideband   & 400 & 390.625\footnotemark[1]\\
Beamformer & 400 & 390.625\footnotemark[2]\\
8k Wideband & 400 & 48.8\footnotemark[3]\\
HI Galaxy Clusters & 400 / 16 = 25  & 25000 / 4096 = 6.1  \\
HI Large Galaxies  & 400 / 32 = 12.5 & 12500 / 4096= 3.0517 \\
HI Galaxies / Maser Search  & 400 / 64 = 6.25\footnotemark[4]  &  6250 / 4096 = 1.525879 \\
Maser Monitoring & 400 / 256 = 1.5625 & 1562.5 / 4096 = 0.3814697\\
\hline

\end{tabular}
\\
$^1$The channel bandwidth is obtained by dividing the IF  bandwidth (400 MHz) by the total number of channels (1024).  Note  that only 256 MHz of the IF bandwidth is usable due to RF filtering.\\
$^2$Similar to wideband mode, single boresight beam only.\\
$^3$The IF bandwidth (400 MHz) is divided into 8192 channels.\\
$^4$The usable Bandwidth is slightly less\\
\label{table:modes}
\end{table*}


%\section{Design Drivers}
\section{Design Drivers}
\label{sec:design}

\noindent
KAT-7 was designed primarily as an engineering test-bed and a
risk-mitigation platform for MeerKAT\@. These new technologies were
expected to reduce either capital or running costs when compared to a
more `traditional' radio telescope array. The novel parts included:


\begin{itemize}
\item On-site manufacture of single-piece reflectors made of a
  composite material using a vacuum infusion process. This had an
  embedded fine wire mesh to act as the radio reflective surface.
\item A single motor drive per axis with an anti-backlash mechanism
  and the ball-screw concept for elevation.
\item Stirling cycle cooling to cryogenic temperatures (80\,K) for the
  receiver.
\item RF \textcolor{blue}{over} fibre for data transport.
\item Reconfigurable Open Architecture Computing Hardware) (ROACH)
  boards for all stages of correlation and beamforming.
\item Control using the KATCP (Karoo Arrat Telescope Control Protocol)
 protocol (see section~\ref{sec:KATCP}).
\item Digital data transport in SPEAD (Streaming Protocol for
 Exchanging Astronomical Data) packets (see section~\ref{sec:SPEAD}).
\end{itemize}

We decided to employ 12-m diameter prime focus dishes which, at that time,
were aligned with results from SKA optimization studies
\citep{strawman}. These were designed to provide accurate
pointing (\textless 25\arcsec) and aperture efficiency \textgreater
50\% across the frequency range of 500~MHz--10~GHz. This range
was chosen to be aligned with the MeerKAT goals then in mind. Because
of the dangers of lightning strikes on the composite dish, lightning rods were
included.\textcolor{blue}{These would not be so important on a solid metal
dish as the huge currents induced would have a lager area to conduct them to the
ground.}

\textcolor{blue}{
Science goals for the seven element array were considered secondary to the
design and construction of the array elements. The original scientific 
considerations for the antenna layout and backend frequency coverage are 
considered in section~\ref{sec:freq}.}
%\section{Choice of Frequency}
\section{Choice of Frequency}
\label{sec:freq}
%based on stuff by Adriaan

\noindent
One particular scientific niche that was identified for a small-scale
interferometer in the southern hemisphere was to make high-sensitivity
observations of low surface-brightness emission from extended neutral
hydrogen in the nearby Sculptor cluster of galaxies. The
interferometer's response at 1.4\,GHz needed to be sufficient to
resolve structures on angular scales up to 24\arcmin\ (one third of
the antenna primary beam). This set the shortest baseline length at
30\,m. The positions of the six remaining antennas were selected to
optimize the interferometer response for 4-hour observations,
resulting in a randomized distribution with longest baseline of
180\,m.

The receiver frequency range was chosen to be 1200--1950\,MHz in order
to avoid potential interference from terrestrial GSM and
aeroplanes. The 1.63:1 bandwidth ratio was still feasible with
corrugated horn feeds. This frequency range also gave the possibility
of joining in with standard 18-cm VLBI observations of OH masers and
continuum sources.

%\section{Antenna and Optics}
\section{Antenna and Optics}
\label{sec:ant}


\noindent
\textcolor{blue}{
To confirm the suitability of the composite antennas for MeerKAT and SKA
it was necessary to show that they had good aperture efficiency at the
highest frequency being considered, and that this lightweight structure
could be controlled in standard observing conditions.}

To achieve aperture efficiency of 50\% at 10\,GHz with standard
horn feeds \citep{olver}, the unweighted
small-scale reflector accuracy was required to be better than 1.5-mm
r.m.s. This was achieved by combining theodolite
measurements of the completed antennas with finite-element analysis
for the worst-case load conditions. Follow-up measurements with radio
holography \citep{scott} using a number of geostationary satellite
beacons around 11.7\,GHz confirmed these results.

The antennas achieve blind pointing accuracy of 25\arcsec\ r.m.s under
all but the most extreme operating conditions, with a jitter of no
more than 5\arcsec\ over time scales of seconds. This is achieved at
tracking rates up to 0.05\degr/sec, across the elevation range. Below
2\,GHz the impact on measurements across the antenna beam is no more
than 1\% , and increases to a maximum of 5.5\% at 10\,GHz. Optical
pointing results show that referenced pointing \citep{rupen}, which
involves regularly checking the pointing against a pointing
calibration source near the field being observed, may allow accuracies
of better than 10\arcsec\ to be achieved, which would reduce the
impact to 2\%.
The focal ratio was chosen to be 0.38, which is close to optimal for
single pixel feeds using a conical horn feed \citep{rudge}.

\begin{figure}
\includegraphics[width=\columnwidth]{figs/f3.eps}
\caption{Two of the KAT-7 dishes viewed side-on. Note the lightning rods at the apex and near the focus, the backing structure of steel beams with circular holes, the small counterweight on the right and the sun shield on the left hand side of the pedestal. }
\label{fig:CAD}
\end{figure}



\begin{table}
\caption{Dish specification}
\begin{tabular}{lc}
\textbf{Parameter} & \textbf{Value}\\
\hline
Pointing Accuracy & 25\arcsec \\
Surface Accuracy & 1.5~mm r.m.s.~(spec.)\\
 & 1.0~mm r.m.s.~(goal) \\
Specified Upper Frequency Limit & 10 GHz \\
Wind (Operational)  & 36 km/h \\
Wind (Marginal Operation) km/h & 45 km/h \\
Wind (Drive to Stow) & 55 km/h \\
Wind (Survival)  & 160 km/h \\
Azimuth Rotation slew speed  & 2\degr /s \\
Azimuth limits &  $-$175\degr, +285 \degr \\
Elevation slew speed & 1\degr /s \\
Elevation limits & 0\degr, 90\degr \\
Diameter &  12 m \\
Focal ratio f/D & 0.38 \\
Lowest Natural Frequency & 3 Hz \\
Feed/Cryo Mass  & 75 kg \\
Mount Type & Alt--Az Prime Focus \\
\end{tabular}
\label{table:ant}
\end{table}

The picture of two dishes is shown in figure~\ref{fig:CAD} and dish
specifications are shown in table~\ref{table:ant}.  \textcolor{blue}{
  Primary beam measurements at L-band were initially done for total
  intensity mapping by raster scans across very bright radio sources
  (in particular Hercules~A, Orion~A, Taurus~A and Virgo~A). These
  were found to be consistent with modelling done by EMSS, the company
  doing the receiver systems and optical design. Later the beam shape
  was measured more accurately using a full polarization
  holography-like technique on the same bright sources (with some
  dishes tracking while others made spiral scans around the
  source). The broad features were the same, but the spiral scan mode
  showed details of the secondary lobes and the instrumental
  polarization in the primary beam.}

 \subsubsection{Why composites?}

 There are three main possible choices for a large dish antenna; a
 conventional steel structure based on panels, a lighter but possibly
 more expensive aluminium structure with panels, or a composite dish
 with metal backing for rigidity. For KAT-7 the choice was made for a
 composite for the following reasons:

\begin{itemize}
\item The dish front surface is constructed as a single unit, which
  gives it inherent stiffness (more efficient structure), something
  that is lost when the dish is constructed from loose panels.  This
  is an important aspect which reduces the mass while keeping the
  stiffness of the dish.
\item When constructed as a single unit there are no discontinuities
  in the reflective surface.
\item When constructed as a single unit there is no need to set up and
  align panels on a backing structure that is off the ground, which
  could prove to be time consuming and would be required for every
  dish.  Setting up of the mould is only required once, but must be
  done accurately.
\item Combining the composite dish surface with the steel rib and web
  backing gives a cost-effective solution.
\item An infusion process can be used, which is a tried-and-tested
  technology on such large structures.
\item Tooling is relatively inexpensive.

\end{itemize}

\textcolor{blue}{ It is also very convenient that the thermal
  expansion coefficients of the composite dish front and the steel
  backing are very close to each other, giving low thermal
  loading. Thermal loading is potentially a large problem for pointing
  and surface accuracy, given the large temperature changes in the
  semi-desert climate of the Karoo.} 
In addition the lightweight
nature of the dish meant that the counterweight needed would be light
and that the motors could be low powered and still achieve good slew
speeds.  \textcolor{blue}{ The main drawbacks for a composite laminate
  dish are that the mould accuracy must be higher than that of the
  desired surface and that the layers of the laminate must be done in
  a balanced and symmetrical way to reduce the inherently anisotropic
  nature of the fibre mats. For balance any laminae must be placed
  symmetrically around a centre line, and for symmetry any laminae at
  layer (\textit{n}) must be have an identical ply at layer
  (\textit{-n}) from the middle of the layers. For example a minimal
  set would be weave layers in the sequence (0,90), (45,-45), (45,-45)
  and (0,90)\degr\ . This symmetry is straightforward for a
  circularly symmetrical dish, but more complicated for other shapes.}

%\section{RF and IF chain}
\section{RF and IF chain}
\label{sec:RF}

\subsection{Feed}

\noindent
The feed provides native horizontal and vertical linear
polarization. It was decided not to convert to circular polarization
in the receiver. To add a polarizer would have increased the
complexity of the receiver, which was considered a risk. Adding the
required components (90\degr\ hybrid and isolators preceding the
low-noise amplifiers) inside the 80\,K cryostat with a single stage
Stirling cooler would increase system noise by at least 5\%.  A linear
polarization response could be calibrated sufficiently well to meet
the 30-dB linear and circular polarization purity required for
e.g.~measurements of Zeeman splitting; \textcolor{blue}{making
  a feed for a large fractional bandwidth (\textgreater~30\%) while
  retaining accurate circular polarization is hard.}

\subsection {RFE Architecture}

\noindent
The Radio frequency Front End (RFE) architecture of KAT-7 is shown in 
Figure~\ref{fig:block}.
The design was driven by the following design decisions:

\begin{itemize}
\item Reduce complexity of the RFE at the antenna focus to reduce
  self-generated RFI to the maximum extent possible, in preference to
  mitigating RFI later.
\item Reduce complexity of the RFE at the antenna location and move
  the down-converter and clock distribution to the same location as
  the digitiser.
\item Test the concept of mitigating radio frequency interference
  (RFI) by moving the digitiser away from the antenna location to a
  shielded container located about 6\,km away, with the Losberg
  mountain providing a very high order of RFI shielding.
\end{itemize}  



\begin{figure*}
\includegraphics[width=\textwidth]{figs/f4.eps}
\caption{Signal chain from receiver to Digital Back End(DBE). Signals
  arrive at the horn and go through the OrthoMode Transducer (OMT) to
  the Low Noise Amplifier (LNA). Calibration from a noise generator
  can be injected. After further amplification and filtering they
  descend to the pedestal where they are again amplified. The signal
  level is adjusted by attenuators and the signals is then filtered
  and used to modulate an optical signal. This is transmitted over
  optical fibre and converted back to an RF signal in the Losberg
  facility. This then is mixed with local oscillators and digitally
  sampled for the Digital Back End}
\label{fig:block}
\end{figure*}



\begin{figure*}
\includegraphics[width=\textwidth]{figs/f5.eps}
\caption{An illustration of the down-conversion and sampling
  procedure. On the left side is the signal chain (top to bottom) with
  bandpass filters BF1-BF3, local oscillators LO1 and LO2 and the
  Analogue to Digital Converter (ADC). $f_{LO1}$ and $f_{LO2}$
  represent frequencies of the local oscillators. On the right are the
  spectra at the various stages. The top spectrum shows the signal as
  it arrives at Losberg, the middle spectrum shows the signal (and its
  alias) after mixing with LO1, while the bottom spectrum show the
  signal (and alias) after mixing with LO2, just before it is sampled
  by the ADC\@. N1 to N5 are the Nyquist zones. }
\label{fig:mixing}
\end{figure*}


 
\begin{figure*}
\includegraphics[width=\textwidth]{figs/f6.eps}
\caption{Detail of the last stage of the down-conversion procedure,
  showing the last (fixed) local oscillator and possible aliasing. The
  bandpass filter (BPF3) is designed to reduce that aliasing to a
  minimum before it enters the ADC, so that it does not appear in the
  sampled data. N1-N3 represent different Nyquist zones}
\label{fig:detail}
\end{figure*}

It is a double heterodyne system involving fixed and variable
local-oscillator (LO) signals that translate an RF (in the frequency
band of 1.2--1.95\,GHz) to an IF signal (in the band of 72--328\,MHz)
compatible with the digital samplers.  The first local oscillator
(LO1) can be set in the range 5.5--6.1\,GHz while the second (LO2) is
fixed at 4.0\,GHz.  The mixing scheme is shown in
Figure~\ref{fig:mixing}.  Figure~\ref{fig:detail} gives details of the
last stage of down-conversion and suppression of aliasing. `BPF's are
bandpass filters and `ADC's are analogue to digital converters.





\subsection{Noise Diodes}

%\textbf{XXXX SAY WHY WE HAVE A COUPLER *AND* A PIN XXXX}

\noindent
Each receiver includes two stabilized noise diodes to permit online
monitoring of system noise and receiver gain. The period and the duty
cycle of the noise system can be software selected, as well as having
a simple on/off mode. The signal can either be added to the received
(`sky') signal via an antenna that is integrated into the feed horn
(`the pin') or via 3-dB couplers that are installed between the
OrthoMode Transducer (OMT) ports and the low noise amplifiers (`the
coupler').  The `pin' signal level is approximately equivalent to the
system temperature (${T_{sys}}$) while the `coupler' level is
approximately $\frac{T_{sys}}{10}$. Using the `coupler' it is possible
to calibrate total-power measurements to an absolute scale with
accuracy better than 10\%.  The `pin' injection was designed to allow
delay and phase calibration between the orthogonal linear polarization
channels while the antennas were operating in single-dish mode, and
has been rarely used since the array was functional.  The receivers
were designed to permit observations of the Sun. However, no `high'
noise diode is provided to allow such observations to be calibrated.


\subsection{Down-conversion and Digitisation}

\noindent
To reduce the risk of RFI at the antennas, the analogue RF signal is
transported via fibre-optic cables to a central facility
\textcolor{blue}{behind Losberg mountain}. The signals are then
down-converted using a common LO, after which they are sampled in
baseband (i.e.~first Nyquist zone). The nominal signal is sampled with
3 bits (r.m.s.~$\approx 3$ levels) using 8-bit samplers. Together with
the 27\,dB of headroom to the 1-dB compression point provided by the
end-to-end analogue line-up, this provides the high dynamic range that
is required to limit the response to strong interfering signals. Even
on occasions when navigation satellites traverse the antenna beam,
gain compression remains at or below the 1\% level so intermodulation
products remain undetectable.

The analogue chain was designed to limit instrumental phase
instability to less than $8\degr$ r.m.s.~over 30-minute time scales. A
GPS-disciplined Rubidium maser clock is employed as frequency
reference to which the LOs are phase-locked. This allows the system to
achieve adequate phase stability to limit VLBI coherence loss to
20\% at frequencies up to 2\,GHz.

%\section{Correlator}
\section{Correlator}
\label{sec:corr}

\noindent
The correlator modes are noted in Table~\ref{table:modes}.  The
correlator is based on the so-called `FX' architecture (Fourier
transform `F' followed by Cross-correlation `X') with Hann-windowed
polyphase filter banks (PFBs).

The spectral-line modes are limited to a single band per observation,
which is a minor inconvenience for observations of OH maser lines. The
correlator always computes \textcolor{blue}{all four complex
  polarization products} for all baselines (including
autocorrelations) since, with linearly-polarized feeds, total-power
measurements are easily corrupted by linear polarization and therefore
polarization calibration is required to achieve the highest possible
fidelity total-power imaging.

The correlator performs all internal computations with sufficient
numerical precision to support a spectral dynamic range of 43\,dB.

The Digital Back-End (DBE) also provides a coherently-summed
beamformer in parallel with a wide-bandwidth correlator. The
beamformer voltages are channelized to coincide with the correlator's
390\,kHz-wide channels. This correlator output is used by the Science
Processing Team's online pipeline to keep the
array phased-up during the course of the observation. The Science
Processor also provides the facility to re-synthesize and re-sample the
channelized voltages to the bandwidths required for the observation.


\begin{figure*}
\includegraphics[width=\textwidth]{figs/f7.eps}
\caption{The KAT-7 Correlator. The signal chain (in black) for two of
  the telescopes is shown, as well of the control and monitoring (in
  red). The F-engines do the Fourier transform and X-engines do the
  correlations. For $n$ antennas there are $\frac{n(n+1)}{2}$
  correlation products if you include all autocorrelations.}
\label{fig:correlatorbd}
\end{figure*}


The beamformer's output may be used for VLBI recording or for
performing coherent de-dispersion of pulsar signals.
Figure~\ref{fig:correlatorbd} shows a simplified block diagram of
KAT-7's prototype correlator, consisting of three main components:

\begin{itemize}
\item The \textit{F-engines} which channelise the incoming data
  streams into spectral components.
\item The \textit{X-engines} which multiply and accumulate every
  product pair.
\item A commodity, off-the-shelf network switch to interconnect these
  boards.
\end{itemize}


KAT-7 required additional features to turn the basic correlator into a
more fully-featured digital back-end, most notably beamforming (for
tied-array operation) as well as additional operations in the
correlator itself such as fringe rotation and delay compensation, fast
readout speeds and a standardised output protocol. Hardware changes
were also required: The performance of the existing digitisers within
the CASPER\footnote{\url{https://casper.berkeley.edu}} collaboration
were found to be inadequate for KAT-7, and there was a desire for
KAT-7 to standardise on a single, general-purpose hardware processing
platform that could both digitise and process the signals.

In its current form, the KAT-7 DBE consists of 16 ROACH boards, eight of
which are configured as F-engines and the remaining eight as X-engines. It
is a real-time, full-Stokes, asynchronous packetized design with 16
inputs, processing 400\,MHz of RF bandwidth (only 256\,MHz of which
contains useful analogue signal because of the anti-aliasing filters
introduced in the RF chain (see Figure~\ref{fig:detail}).
ROACH boards can optionally host ADCs for digitising analogue signals
and the F-engines in KAT-7 are equipped with KAT ADCs to digitise the
down-converted IF signal at 800\,MSample/s with an 8-bit resolution. A
20-port 10\,GbE network switch is used to interconnect all the
processing boards.

While the F-engines and their ADCs are synchronously clocked from a
GPS-disciplined rubidium common clock source, the X-engines are purely
compute nodes that operate from their own, asynchronous,
clocks. Packets are timestamped in the F-engines and these headers are
interrogated by subsequent processors for data identification and
re-alignment. These \textcolor{blue}{X-engines} could easily be
replaced by ordinary CPUs and GPUs when technology advances enable
them to cope with the data rates presented.

The fringe rotation and delay compensation were implemented within the
F-engines by employing a combination of time-domain and
frequency-domain processing, and the system currently allows for
phased-tracked wideband (continuum) and spectral-line observations. It
does not allow online Doppler tracking.

Frequency-domain beamforming is implemented by tapping a copy of the
channelised data destined for the X-engines, and presenting it to a
co-located \textit{B-engine} which performs beam steering and
summation. The existing F-engines are thus shared by the X- and
B-engines. Development time and risk were reduced by leveraging the
existing delay-compensation software. This B-engine is not resource
hungry, needing only a single complex multiplier and adder to sum the
already serialised data, along with a lookup table for the steering
coefficients. This makes it possible to create additional beams with
modest incremental hardware costs.

All systems are being designed to be directly transportable and
scalable up to larger arrays such as MeerKAT and the SKA.

%\section{Control}
\section{Control}
\label{sec:control}

\subsection{The KAT Control Protocol: KATCP}
\label{sec:KATCP}

\noindent
The MeerKAT telescope contains a large number of devices that require
control and monitoring, usually both. Most of these devices are
bespoke, meaning that they were constructed either within the project
or by a contractor to the specifications defined by the project.  As a
consequence we had the luxury of being able to select a communications
design strategy, architecture and protocol suite used for the control
and monitoring of the telescope.

A significant design decision was to confine low-latency, real-time
control functions to a small set of components. In other words the
control system as a whole does not operate on a low-latency basis ---
instead, time critical commands are sent out well in advance and are
scheduled by a smaller real-time component close to the limited number
of elements requiring such control. The system distributes a clock
signal to subordinate nodes.  This decision makes it possible to use
computer systems running conventional (not hard-real-time) operating
systems that communicate using common, non-deterministic interconnects
and protocols. In particular it was possible to select TCP/IP over
Ethernet for transport: these are well-established, pervasive and
inexpensive lower communications layers.

The only locally developed part of the control network is the
application control protocol, KATCP, running on top
of TCP/IP\@. KATCP borrows from established, classical Internet
protocols such as SMTP, FTP and POP in that is an application protocol
which is easily read by humans; each line of text is a protocol
message. The protocol is not too different from a
command prompt or shell interface, where each input line is also a
command. However, unlike command line utility output, the protocol
replies have a regular, well-defined structure\footnote{See
\url{https://casper.berkeley.edu/wiki/KATCP}
 and \url{https://pythonhosted.org/katcp}.
}.


\subsection{Streaming Protocol for Exchanging Astronomical Data: SPEAD}

\noindent
In analogy with the control protocol, we unified the data exchange
protocol wherever practicable. Hence the correlator internals,
correlator output stage, and science processor all share a common
data-exchange format called
SPEAD\footnote{The specification document and a full description can
be found at \url{https://casper.berkeley.edu/wiki/SPEAD}}. 


\textcolor{blue}{
SPEAD is a one-way, best-effort, self-describing protocol, containing
both machine and human-readable descriptions. This allows receiver
processes to automatically unpack data, enabling application designers
to concern themselves with algorithms rather than with data
exchange. It also allows users to comment their data-streams so that
stored data can easily be reinterpreted at a later date without
additional documentation, or to easily debug data exchanges.}

Other data protocols such as the VLBI Data Interchange Format, 
VDIF~\footnote{\url{http://www.vlbi.org/vdif/}} were considered, 
\textcolor{blue}{but it was decided
that they were not sufficiently self-describing or flexible.
}
SPEAD was
developed in collaboration with the PAPER (Precision Array for Probing
the Epoch of Reionization; \citealt{parsons}) team as part of the
broader CASPER collaboration and is available under an open-source
(GPL) license. SPEAD can be used for on-the-wire exchange, for on-disk
storage or for piping data between application processes within a
single compute node. 
This protocol is implemented on ROACH boards, Central Processing
units (CPUs) and Graphical Processing Unit (GPUs).

SPEAD is in essence optimised for the transmission of arrays of data
from one location to another. In particular we use SPEAD extensively
for the movement of Python Numpy arrays (Figure~\ref{fig:SPEAD}).  It
is designed to allow exchange of arbitrary data structures and aims to keep
receivers' copies fresh by propagating any changes
to the receiver as these changes occur on the transmitter.
\textcolor{blue}{
The metadata and variable descriptors are injected into the primary data stream.
Such changes will occur if the correlator is started, or changes mode
or the frequency of a local oscillator is altered.
Receiver processes keep state information, so
the latest values of all variables within the structure are always
available, even if only the changed subset is transmitted.
If these dynamic features are not required, a lightweight and faster
static implementation can be made.}

\label{sec:SPEAD}
\begin{figure}
\includegraphics[width=\columnwidth]{figs/f8.eps} 
\caption{SPEAD exchanges between a range of platforms. Once the data has been sampled and Fourier transformed in the F-engine all the data is handled internally as SPEAD packets or as Numpy arrays.}
\label{fig:SPEAD}
\end{figure}

As a one-way transport, and because it operates over UDP on Ethernet
networks, it is easy to support multicast transparently, allowing
multiple devices to subscribe to the same data-streams.
\textcolor{blue}{This allows
real-time data inspection and plotting by subscribing to a subset of the 
data. As UDP does not check for packet delivery SPEAD was made to tolerate
some packet loss, but cannot request a lost or corrupted packet.
Buffering helps by allowing for packets being received out of sequence.}



%%%%%%%%%%%%%%%%%%%%
\section{First Image with KAT-7}
\label{sec:images}

\noindent
The first tests of the completed KAT-7 system were undertaken in 2011
August using the Wideband continuum mode (See
Table~\ref{table:modes}). The specific combination of sensitivity,
angular resolution, $\approx$ 3(4)$\arcmin$ at 1822(1328)\,MHz, and
field-of-view ($\approx 1\degr$) of KAT-7 require extended southern
sources for use as imaging targets to use for testing this mode.  One
such source is \mbox{PKS 1610-60} which has an integrated flux density of
43\,Jy at 1822\,MHz, and an extent of 30\arcmin\ in right ascension and
4\arcmin\ in declination \citep{christiansen}.  This source was
observed with KAT-7 for 10 hours with both polarizations at a
frequency of 1822\,MHz. The KAT-7 correlator outputs the data into an
observatory-specific format based on HDF5.  After conversion to a
standard Measurement Set the data are calibrated and imaged in CASA
\citep{mcmullin}. The final total-intensity image can be found in
Figure~\ref{fig:pks1610}. The dynamic range (ratio of the peak image
intensity to the r.m.s.~image noise) of this map is about 200:1.


\begin{figure}
\includegraphics[width=\columnwidth]{figs/f9.eps} 
\caption{Synthesis image of \mbox{PKS 1610-60} region containing the more
  prominent bent-double radio galaxy \mbox{PKS 1610-60.8} (below) and the
  smaller radio galaxy \mbox{PKS 1610-60.5} (above).}
\label{fig:pks1610}
\end{figure}



\section{Radio Continuum Imaging and Monitoring}
\label{sec:continuum}

\noindent
A sample of calibrators ($-90\degr<\delta<30\degr$ ) are continuously
being monitored and identified as possible short-spacings flux-density
calibrators \citep{kassaye}. The aim of this monitoring campaign is
also to image potential new calibrators with short spacings and to
assess their long-term variability. Several sources that are suitable
as point-source calibrators for interferometers with larger dishes,
and consequently smaller fields-of-view (such as e.g. \mbox{J0943-081}
and \mbox{J2326-4027}) have proven unsuitable for KAT-7 as they have
nearby contaminating sources within the primary beam.
%re-written -Tony
In general we have rejected candidate calibrators that are found to
have a contaminating source above 10\% of their peak
\textcolor{blue}{brightness}, or where the integrated flux density
(estimated using the shortest baseline) is more than 10\% greater than
the maximum \textcolor{blue}{flux density per beam}
%
% Most placed on the sky will have a few hundred milliJansky spread
% over a few sources within the KAT-7 primary beam at 1.2-1.95\,GHz.
 \textcolor{blue}{Given the size of the KAT-7 primary beam at
  1.2 to 1.95\,GHz, most maps have few hundred milliJansky spread over a
  number of sources brighter than 1mJy/beam. This is consistent source
  number counts from the FIRST \citep{white} and  ATLBS \citep{sub} surveys.}  With the modest
angular resolution of KAT-7 we would expect our images to be limited
by confusion noise rather than thermal noise in Stokes
\emph{I}. \textcolor{blue}{For most places on the sky} we hit a
continuum confusion limit of about 1(2) mJy/beam at 1822(1328)\,MHz
(based on no more than one source per five beams), and making high
dynamic range continuum images where confusion is so high has proven
difficult. \textcolor{blue}{The confusion limit is even higher along
  the galactic plane.}

\textcolor{blue}{ From a full-polarization analysis, the polarization
  fraction of $8.4\pm0.6 \%$ was seen for 3C286 and $8.1\pm0.6 \%$ for
  3C138. These values are consistent with those determined by
  \citet{perley}.  Polarization was calibrated using full-Stokes
  observations of unpolarized radio sources such as \mbox{PKS
    1934-638} to determine instrumental leakage (of total intensity
  into polarization), and sources with high linear polarization at
  L-band over a large range of parallactic angles to measure delay and
  phase offsets between horizontal and vertical dipoles.  Typical
  leakages within the half-power beam width were of the order of 3\%
  in the upper band (1822\,MHz) and 3--6\% in the lower band
  (1322\,MHz), so second-order correction effects were negligible.}

\textcolor{blue}{ Although the dynamic range in continuum total
  intensity is limited by confusion after about 8 hours, this is not
  the case for polarization or spectral-line observations. We do
  however need the instrumental polarization corrections for high
  dynamic range spectral line observations, such as in OH maser
  observations.}


\section{Commissioning Science with KAT-7}
\label{sec:comm}

\noindent
KAT-7 went through in a science commissioning phase.  The requirements
for commissioning science projects are (i) that they test the
performance of various aspects of commissioning (e.g.~correlator
modes, data reduction pipeline, RFI mitigation), and (ii) they
endeavour to contribute new and original scientific results.  These
projects touch on a broad range of scientific interest, including, but
not limited to, topics identified as MeerKAT Key Science Projects.
Projects include:

\begin{itemize}
\item Monitoring variable continuum sources (selected from ATEL
  announcements).
\item Imaging extended, diffuse continuum emission.
\item Timing strong southern pulsars.
\item Conducting very long baseline interferometry (VLBI).
\item Measuring HI spectral line kinematics and low surface brightness emission in nearby galaxies and clusters.
\item Monitoring OH maser emission.
\item Mapping large scale polarization.
\item Implementing techniques for HI intensity mapping.
\end{itemize}

\noindent
We outline several of these completed and ongoing science
commissioning projects in the following subsections.  We highlight
their scientific importance, and their contributions to KAT-7
commissioning and preparations for MeerKAT.

\subsection{Transient sources}

\noindent

Many transient sources 
%(e.g.~PKS~1424-418, Circinus~X-1, PKS~2356-502)
have been monitored with KAT-7 ranging from neutron-star 
and black-hole X-ray binaries, blazars, galactic 
gamma-ray binaries, and novae.
Long-term monitoring of Circinus X-1, a bright and highly variable
X-ray binary, was carried out using KAT- 7 at 1.9\,GHz and the HartRAO
radio telescope at 4.8 and 8.5\,GHz by \citealt{armstrong}, and is
the first published scientific paper using KAT-7 data, and a good
example of multi-wavelength collaboration between South African
instruments. The observations confirm a return to strong radio flaring
first observed in 1970's that had been suppressed for two
decades. Figure~\ref{fig:2F_OBS} shows this long-term behaviour.

\begin{figure}
\includegraphics[width=\columnwidth]{figs/f10.eps}
\caption{A radio light curve of Circinus~X-1 obtained with KAT-7 (1.9\,GHz,
red crosses) and HartRAO (4.8\,GHz, black circles and 8.5\,GHz, blue stars). 
The X-ray binary system is clearly flaring to Jansky levels again. The times in 
which KAT-7 is operational are shown in green. The vertical bars at each
point represent the measurement noise, while the horizontal bars represent
the observation length.}
\label{fig:2F_OBS}
\end{figure}

Simultaneous observations of the black hole candidate Swift
\mbox{J1745-26} with KAT-7, the Very Large Array (VLA), and the Australia
Telescope Compact Array (ATCA) radio interferometers reveal a `failed outburst'
event, illuminating details of the complex processes of accretion and
feedback in black holes \citep{curran}.

The blazar \mbox{PKS 1424-418} was observed as follow-up to ATEL4770
(Figure~\ref{fig:atel4770}).  Similar observations have been carried
out using the blazar source \mbox{PKS 1510-089} \citep{oozeer},
whereby at least one flaring has been successfully recovered. Another
weak transient source, \mbox{PKS 2356-502}, was observed in 2012 May
and continues to be monitored (Figure~\ref{fig:2326}).

\begin{figure}
\includegraphics[width=\columnwidth]{figs/f11.eps}
\caption{\mbox{PKS 1424-418} imaged after ATEL4770. Contours start at $\pm$
  10\,mJy/beam and go in steps of a factor of 4. Negative contours are
  shown dashed and the synthesized beam is shown in the top left
  corner. No primary beam correction has been applied.}
\label{fig:atel4770}
\end{figure}

\begin{figure}
\includegraphics[width=\columnwidth]{figs/f12.eps}
\caption{\mbox{PKS 2356-502}. Contours start at $\pm$ 2.5\,mJy/beam and go in
  steps of a factor of 4. Negative contours are shown dashed and the
  synthesized beam is shown in the top left corner. 
  The FWHM of the primary beam is $1.0\deg$. No primary beam
  correction has been applied. 
}
\label{fig:2326}
\end{figure}




\subsection{Diffuse radio emission}

\noindent
With the availability of short spacings (26--185\,m, see
Table~\ref{table:array}), KAT-7 is an ideal instrument to detect
extended diffuse radio emission especially in galaxy clusters hosting
halos and relics. For example, Abell~3667 was observed with KAT-7 and
the nature of the diffuse emission (two large-scale relics) was confirmed
(see \citealt{riseley}). 
Figure~\ref{fig:aco} shows the continuum image of
Abell~3667 at 1822\,MHz.


\begin{figure}
\includegraphics[width=\columnwidth]{figs/f13.eps}
\caption{Total intensity map of Abell 3667 at 1826\,MHz. Image noise
  is 1.30\,mJy/beam. Contours mark [-1,1,2,3,4,6,8,12,16,32,64,128]
  $\mathrm{\times \sigma_{rms}}$. The synthesised beam size (empty circle bottom left) is
  $\mathrm{184 \times 150 \arcsec}$.
  The FWHM of the primary beam is $\mathrm{1.0\deg}$.
  }
\label{fig:aco}
\end{figure}


figure \subsection{Pulsars}

\noindent


KAT-7 high-time resolution capability was first demonstrated through a
single-dish observation of the Vela pulsar (\mbox{J0835-4510}) in late
2012. Here Nyquist-sampled raw-voltage data was written to disk and
processed offline using a custom pipeline which incorporates
{\small\textsc{dspsr}}\footnote{\url{http://dspsr.sourceforge.net/}}
and
{\small\textsc{psrchive}}\footnote{\url{http://psrchive.sourceforge.net/}}
tools \citep{hotan}. Semi-regular, dual-polarization observations of
bright pulsars ($\gtrsim 10$~mJy) with the KAT-7 beamformer have since
become standard practice. Single-pulse and folded searches have been
performed successfully on a number of sources. A series of short
pulsar timing campaigns have also been undertaken, predominantly
focusing on \mbox{J0835-4510} due to its low period ($\sim$89.3~ms)
and high flux density (1.1~Jy at 1400\,MHz;
\citealt{bf74}). Figure~\ref{fig:pulsarspec} shows a comparison
between the timing residuals obtained in 2014 November for
\mbox{J0835-4510} using HartRAO and KAT-7 data.  \textcolor{blue}{
  Figure~\ref{fig:pulsarfold} shows an observation of the bright pulsar,
  \mbox{J0835-4510}. It was reduced and the plot was made using \textsc{psrchive}
  \citep{hotan} software.}

\begin{figure}
\includegraphics[width=\columnwidth]{figs/f14.eps} 
\caption{Phase residuals for Vela (\mbox{J0835-4510}), as produced by
  the pulsar timing package \emph{tempo2} \citep{hobbs}. Those from
  the 26~m HartRAO telescope (observed almost daily) are plotted in
  red, whilst the blue points show observations made with the KAT-7
  beamformer.  The plot confirms that the timing is consistent between
  the two radio telescopes.}
\label{fig:pulsarspec}
\end{figure}

\begin{figure}
\includegraphics[width=\columnwidth]{figs/f15.eps} 
\caption{Observation of the bright pulsar, \mbox{J0835-4510}, done
  with KAT-7. These were taken for several minutes and this plot
  presents the results after folding at the pulsar period of
  89~milliseconds. The horizontal axis show the pulse phase and the
  lower figure shows power as a function of frequency, with integrated
  power over the band plotted above, while the upper figure shows
  pulse power as function of time. The darker colour denotes higher
  power.}
\label{fig:pulsarfold}
\end{figure}


\subsection{VLBI}

\noindent
Pathfinder VLBI observations are critical to the establishment of the
African VLBI Network (AVN; \citealt{gaylard}) in advance of SKA1.
Although KAT-7 is not equipped with a hydrogen maser, typically
required for VLBI observations, we obtained fringes between a single
KAT-7 antenna and the 26-metre antenna located at HartRAO near
Johannesburg, South Africa. The local GPS-disciplined rubidium time
source was deemed sufficient for this experiment.  A short 30-minute
test observation of the bright source 3C273 was performed on 2010
November 23. This involved HartRAO recording the data to their Mark~V
VLBI system, and KAT-7 recording to disk using the raw-voltage capture
system.

Due to the sampling clock of KAT-7 not being a power of 2, we used a
DDC (digital down converter) to convert the sample rate in order to
produce the resultant 64\,MHz bandwidth, via CUDA code deployed to a
GPU.
 
The data were combined and reduced using in-house python scripts. The
data were then verified using the DiFX correlator package
\citep{Deller}, with strong fringes subsequently detected
(Figure~\ref{fig:vlbifringe}).


\begin{figure}
\includegraphics[width=\columnwidth]{figs/f16.eps} 
\caption{VLBI fringes obtained between KAT-7 and HartRAO using DiFX
  software correlator. The horizontal axis is frequency channel and
  the plots in descending order are amplitude, phase (in degrees) and
  lag.}
\label{fig:vlbifringe}
\end{figure}



\subsection{Spectral Lines: HI and OH observations}

\subsubsection{HI}

\noindent
The receiver on KAT-7 covers the 1420\,MHz transition of neutral
hydrogen (HI) redshifted as far as 1200\,MHz (z=0.184),
although the intrinsic faintness of the
spectral line and the resolution of the telescope make it most
suitable to emission-line studies in the nearby Universe. In
particular, KAT-7 is well suited for the study of nearby southern
objects which have emission on scales larger than 15$\arcmin$,
typically invisible to telescopes such as the VLA that lack the short
baselines. This is well illustrated in the KAT-7 HI observations of
NGC~3109 \citep{Carignan}, where 40\% more HI flux was detected
than in previous VLA measurements (see Figure~\ref{fig:ngc3109}). The
new data also allowed the derivation of the rotation curve 16$\arcmin$
further out than the previous VLA measurements (see Figure 10 of
\citealt{Carignan}). These observations were undertaken as part of
commissioning the 6.25\,MHz spectral-line correlator mode, `HI 
Galaxies' (Table~\ref{table:modes}). They led to the
discovery of a source of antenna-dependent, faint, very narrow
internally-generated RFI that was successfully eliminated by the
insertion of a low-pass filter along the signal path.

The wide field-of-view of KAT-7 and wide-bandwidth spectral-line modes
make it a powerful instrument for mosaicking large areas on the sky,
competitive with ATCA in terms of total observing time for similar
sensitivity.  As part of commissioning the 25\,MHz `HI Galaxy
Clusters' spectral-line correlator mode, several projects
to image the HI in galaxy groups and clusters are currently underway
(e.g.~Antlia Cluster, NGC~4055 Group). In particular, the Antlia
Cluster was observed for 147 hours, in a 7-pointing discrete hexagonal
pattern, covering roughly $4.4\mathrm{deg^2}$ , and reaching 
$0.97 \mathrm{mJy \,beam^{-1}}$
over the $15.5 \mathrm{km \,s^{-1}}$ channels, see Figure~\ref{fig:antlia2} \citealt{hess}.  This deep map detected 30 HI cluster members, 28 of
which were new HI detections, and 20 of which were the first redshift
measurements at any wavelength.  The broad bandwidth coverage, and
lack of bright, dominating sources in the Antlia mosaic led to the
recognition of the `$u=0$' problem in KAT-7 spectral-line data.  In
this case, RFI that exists below the level of the noise in the
visibility data, and would normally be incoherent, adds constructively
when the fringe rate between two antennas is equal to zero.  It is not
a unique phenomenon to KAT-7, but it is particularly a problem on
short baselines.  (See \citealt{hess} for a more complete explanation).

\begin{figure}
\includegraphics[width=\columnwidth]{figs/f17.eps}
\caption{HI mosaic of NGC~3109 and Antlia dwarf superposed on a
 DSS~B image. The green contours are [0.5, 1.07, 2.1, 3.2, 5.4,
 10.7, 21.4, 32.2, 53.6, 75.0, and 96.5] times the peak of 
 $107.2 \mathrm{Jy \,{beam}^{-1} \times km \,s^{-1}}$.
 NGC~3109 ($\mathrm{V_{sys}  = 404 \,km\, s^{-1}}$) is at the
 top and Antlia ($\mathrm{V_{sys} = 360 \,km\,s^{-1}}$) is at the bottom. The two
 background galaxies ESO~499-G037 ($\mathrm{V_{sys}  = 953\, km \,s^{-1}}$) and
 ESO~499-G038 ($\mathrm{V_{sys} = 871\, km \, s^{-1}}$) with its associated HI cloud
 ($\mathrm{V_{sys} = 912 \,km \,s^{-1}}$) are between NGC~3109 and Antlia. The
 synthesized beam is shown in the lower-right corner.}
\label{fig:ngc3109}
\end{figure}

\begin{figure}
\includegraphics[width=\columnwidth]{figs/f18.eps}
\caption{A cut-out of the Antlia Cluster mosaic: color contours
  corresponding to the systemic velocity of the galaxy detected in HI
  emission are overlaid on a WISE 3.4\,$\mu$m image. NGC~3281,
  detected in absorption, is in dashed contours.  Dark blue is
  $\mathrm{<2200 \,km \, s^{-1}}$, cyan is $\mathrm{2200-2800 km
    \,s^{-1}}$, magenta is $\mathrm{2800-3400 \,km \,s^{-1}}$, and red
  is $\mathrm{>3400 \,km\, s^{-1}}$.  HI rich galaxies are detected in
  a ring around the dominant elliptical galaxy, NGC~3268. These KAT-7
  observations had a usable bandwidth sensitivity to HI emission at
  recessional velocities between $\mathrm{1200-4600 \,km \,s^{-1}}$,
  and 30 HI galaxies were detected between $\mathrm{1800-4300 \,km
    \,s^{-1}}$.  For more details see \citealt{hess}}
\label{fig:antlia2}
\end{figure}

\subsubsection{OH}

\noindent 
The frequency range of KAT-7 covers all four ground-state transitions
of hydroxyl: 1612, 1665, 1667 and 1720\,MHz.  Maser emission occurs in
these transitions under a variety of conditions: in massive
star-forming regions (primarily in the main-lines at 1665 and
1667\,MHz \citep{caswell98}, in the shells around AGB stars at
1612\,MHz \citep{sevenster}, and in shocked regions such as supernova
remnants at 1720\,MHz \citep{wardle}.

While KAT-7 does not have the angular resolution to map individual
maser spots, its narrowest spectral mode gives a velocity resolution
of 68\,m/s, enough to resolve narrow maser lines in massive
star-forming regions.  A number of methanol masers in star-forming
regions have been found to exhibit periodic variations
(\citealt{goedhart} and references therein). \citet{green} attempted
to monitor a short-period source in the hydroxyl main-lines and found
a weak indication of periodicity, but the time-series was
undersampled. Monitoring is an ideal niche application for KAT-7 since
it generally has more time available than the Parkes Telescope or
ATCA\@. Six of the known periodic methanol masers have hydroxyl maser
counterparts that can be detected with KAT-7. These sources are
being monitored on a weekly basis at both 1665 and 1667\,MHz, using
interleaved observations at both frequencies in a 13-hour schedule
block.

\begin{figure*}
\includegraphics[width=\textwidth]{figs/f19.eps}
\caption{Masers in \mbox{G331.13-0.24} at 1665\,MHz monitored with KAT-7. The
  underlying greyscale map shows the positions of the continuum
  emission while the coloured contours show the continuum-subtracted
  maser emission. The insets show the spectrum at the indicated
  positions, where the radial velocity scale is $\mathrm{m \, s^{-1}}$ with
  respect to
  the Local Standard of Rest. The synthesized beam is shown as a
  dashed ellipse at the bottom left corner.}
\label{fig:masers}
\end{figure*}

Figure~\ref{fig:masers} shows a pilot observation of the star-forming
region \mbox{G331.13-0.24} at 1665\,MHz. The large field-of-view of KAT-7 led
to the detection of several maser sources as well as several diffuse
HII regions. The positions, velocities and flux densities are
consistent with those from recent Parkes observations
\citep{caswell14}, confirming that the system is performing as
expected.  The observations are dynamic-range-limited and three of the
six fields observed (including \mbox{G331.13-0.24}) are dominated by off-axis
sources, potentially creating errors in the measured flux density of
the target source.  Direction-dependent calibration techniques are
still under investigation \citep{bhat}. Preliminary results from the
fields not affected by these effects show significant variability
(Goedhart et al.~in~prep). Observations of \mbox{G330.89-0.36}, which has a
peak brightness of 824\,Jy\,per beam, were used to measure the spectral
dynamic range and channel isolation. The line-to-line dynamic range
was \textcolor{blue}{31\,dB. A maximum imaging dynamic range of 32\,dB} was
achieved after applying self-calibration and \textsc{clean}ing
interactively.



%=================================================
\iffalse 
\subsection{HI Intensity Mapping}
\noindent
In cosmology, intensity mapping of the HI spectral line is an
observation technique used for surveying the large-scale structure of
the universe (see e.g.~\citealt{bull}). By mapping the distributions
of neutral hydrogen gas in nearby galaxies astronomers are able to
model the dark matter distribution of each galaxy.

KAT-7 records both single-dish (auto-correlation) data as well as
interferometric (cross-correlation) products. Auto-correlation data
can be used for creating the large-scale high-resolution maps, while
the interferometric data allows for the removal of systematic errors
through calibration.

This implementation of combined single dish and interferometric data
has been identified as a possible observation mode for improved
detection of the Baryon Acoustic Oscillation (BAO) feature and KAT-7
may be used as a proof of concept for
future HI intensity mapping surveys (see \citealt{bull}). 
\fi
%============================

\section{Technology Lessons Learned}
\label{sec:lessons}

\noindent
We learned several lessons during the construction and use of the
KAT-7 array.

The lightweight Stirling coolers used for the cryogenic system are
cheaper than conventional Gifford-McMahon (G-M) cycle coolers but need
maintenance (at least) annually. They are not cold enough (80\,K) to
achieve a large cryopumping effect and so the system needs ion pumps
in order to retain a high vacuum and mechanical dampers to reduce the
vibration generated by the Stirling coolers. Both of these sub-systems
need replacement on a regular basis, which involves the use of cranes
and a team of technicians, plus about 2 days on an external pump to
obtain high enough vacuum before the receivers can be used.  The
capital expenditure for Stirling cycle cooling for 3 receivers on all
64 MeerKAT antennas is ZAR 3.3 million less, but the cost of expected
maintenance, parts and electricity bill would be ZAR 1.1 million
more. This means that over a 3-4 year period Stirling cycle coolers
would end up much more expensive to install and operate than G-M
cryogenic coolers. Using the G-M cooling system should also reduce
down-time for MeerKAT antennas and this has not been factored into the
costs.

The composite dishes with embedded metal mesh have large weight
advantages over dishes made of large metal panels which in turn means
that the backing structure can be light and the motors need not be
powerful.  They must be constructed correctly as they cannot be
adjusted or machined after the fact, and this is in turn means that
the fibre weave, temperature, humidity and vacuum need to be very
carefully controlled during fabrication.  Also, the mould needs to be
machined to higher accuracy than is needed for the dish; this is
relatively simple for a circularly symmetrical design, as only one
segment has to be accurately made and then copied.  This is far more
difficult for a design without that symmetry (e.g.~MeerKAT).  Large
single-piece dishes must be made on-site as moving them becomes a
logistical problem.


The ball-screw mechanism for the elevation drive has proven reliable
and simple and it is retained in the MeerKAT design.

The KATCP protocol has shown itself to be flexible and very useful in
testing and commissioning, and can readily be put into scripting form
for routine observations.

SPEAD, although still undergoing active development, has also proven
effective and will be further rolled out for use in MeerKAT.

The correlator architecture, correlator and beamformer based on FPGA
technology and data transfer over a managed network switch, have also
been found to be both flexible and reliable.

For continuum sources we reach a confusion limit of about
0.5--1.0\,mJy (depending on the field) after 8\,hours; and this is
worse in the Galactic plane.  This makes high-dynamic-range imaging
difficult, although dynamic ranges of a few 1000 are regularly
reached. For spectral-line observations this confusion limit is not an
issue.

In summary, KAT-7 is well suited for:

\begin{itemize}
\item Mapping low-surface-brightness continuum emission (e.g.~radio
  halos and relics).
\item Mapping extended ($\geq 10\arcmin$) spectral-line sources, such
  as HI around nearby galaxies and OH emission around star-forming
  regions. The maximum baseline is too short to measure individual
  maser spots, but we can measure their spectra with high frequency
  resolution.
\item Observing the variability of continuum sources $\geq 10$\,mJy at
  1.8\,GHz.
\item Pulsar observations with the beamformer output.
\item VLBI observations with the beamformer output.
\end{itemize}


\section{Acknowledgements}

\noindent
The SKA South Africa project was formed in 2004 and is funded by
the South African Department of Science and Technology (DST), and
administered by the National Research Foundation (NRF).

We would especially like to acknowledge the help of the Adam Deller
and Walter Brisken with running the DiFX correlator package. Jonathan
Quick helped with running the HartRAO VLBI system for the other end of
the VLBI test.

We would also like to thank the unnamed reviewer of this paper who
suggested many improvements.

\bibliography{k7}
%\section{Appendix: Acronyms}
\label{sec:appendix}

%\begin{deluxetable}{ll}
\clearpage
\begin{table}
%\tabletypesize{\footnotesize}
%\tablecaption{Acronyms}
\caption{Acronyms}
%\tablehead{\colhead{Acronym} & \colhead{Meaning}}
%\tablehead{}
%\startdata
\begin{tabular}{ll}
\emph{Acronym} & \emph{Meaning}\\
\hline
ADC & Analogue to Digital Converter\\
AGB & Asymptotic Giant Branch \\
ATCA & Australia Telescope Compact Array\\
ATEL & The Astromomer's Telegram, \url{www.astronomerstelegram.org}\\
AVN & African VLBI Network\\
BAO & Baryon Acoustic Oscillation\\
%BHC & Black Hole Candidate\\
BPF & Band Pass Filter\\
CASPER & Collaboration for Astronomical Signal Processing and Electronics Research \\
CASA & Common Astronomy Software Applications \\
CUDA & Compute Unified Device Architecture\\
DBE & Digital Back End\\
DDC & Digital Down Converter \\
DiFX & Distributed Fourier transform (X)Cross correlation style correlator\\
DMA & Direct Memory Access\\
DSS & Digitized Sky Survey \\
FTP & File Transfer Protocol\\
FWHM & Full Width Half Maximum\\
G-M & Gifford-McMahon \\
GPL & GNU General Public License \\
GPU & Graphical Processing Unit\\
GSM & Global System for Mobile communications \\
DSPSR &  Digital Signal Processing for Pulsars \\
FPGA & Field-Programmable Gate Array \\
FWHM & Full Width Half Maximum \\
FX & Fourier transform (X)Cross correlation style correlator\\
GNSS & Global Navigation Satellite System\\
GPS & Global Positioning System \\
GPU & Graphical Processing Unit \\
%GRBi & Gamma Ray Binary\\
HartRAO & Hartebeesthoek Radio Astronomy Observatory\\
HDF5 & Hierarchical Data Format version 5\\
IF & Intermediate Frequency\\
KATCP & Karoo Array Telescope Control Protocol\\
LNA & Low Noise Amplifier \\
LO & Local Oscillator \\
OMT & Orthomode Transducer\\
PAPER & Precision Array for Probing the Epoch of Reionization \\
PFB & Polyphase Filter Bank \\
POP & Post Office Protocol \\
RDMA & Remote Direct Memory Access \\
RF & Radio Frequency\\
RFE & Radio Frequency End \\
RFI & Radio Frequency Interference \\
ROACH & Reconfigurable Open Architecture Computing Hardware\\ 
r.m.s & root mean square\\
SKA & Square Kilometre Array\\
SMTP & Simple Mail Transfer Protocol\\
SPEAD & Streaming Protocol for Exchanging Astronomical Data \\
TCP/IP & Transmission Control Protocol / Internet Protocol\\
UDP & User Datagram Protocol\\
VLA & Very Large Array \\
VLBI	  & Very Long Baseline Interferometry\\
%XRB & X-ray Binary\\ 
%\enddata
\end{tabular}
%\end{deluxetable}
\end{table}



%\bsp
\label{lastpage}

 \end{document}

