\section{Design Drivers}
\label{sec:design}

\noindent
KAT-7 was designed primarily as an engineering test-bed and a
risk-mitigation platform for MeerKAT\@. These new technologies were
expected to reduce either capital or running costs when compared to a
more `traditional' radio telescope array. The novel parts included:


\begin{itemize}
\item On-site manufacture of single-piece reflectors made of a
  composite material using a vacuum infusion process. This had an
  embedded fine wire mesh to act as the radio reflective surface.
\item A single motor drive per axis with an anti-backlash mechanism
  and the ball-screw concept for elevation.
\item Stirling cycle cooling to cryogenic temperatures (80\,K) for the
  receiver.
\item RF \textcolor{blue}{over} fibre for data transport.
\item Reconfigurable Open Architecture Computing Hardware) (ROACH)
  boards for all stages of correlation and beamforming.
\item Control using the KATCP (Karoo Arrat Telescope Control Protocol)
 protocol (see section~\ref{sec:KATCP}).
\item Digital data transport in SPEAD (Streaming Protocol for
 Exchanging Astronomical Data) packets (see section~\ref{sec:SPEAD}).
\end{itemize}

We decided to employ 12-m diameter prime focus dishes which, at that time,
were aligned with results from SKA optimization studies
\citep{strawman}. These were designed to provide accurate
pointing (\textless 25\arcsec) and aperture efficiency \textgreater
50\% across the frequency range of 500~MHz--10~GHz. This range
was chosen to be aligned with the MeerKAT goals then in mind. Because
of the dangers of lightning strikes on the composite dish, lightning rods were
included.\textcolor{blue}{These would not be so important on a solid metal
dish as the huge currents induced would have a lager area to conduct them to the
ground.}

\textcolor{blue}{
Science goals for the seven element array were considered secondary to the
design and construction of the array elements. The original scientific 
considerations for the antenna layout and backend frequency coverage are 
considered in section~\ref{sec:freq}.}