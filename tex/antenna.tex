\section{Antenna and Optics}
\label{sec:ant}


\noindent
\textcolor{blue}{
To confirm the suitability of the composite antennas for MeerKAT and SKA
it was necessary to show that they had good aperture efficiency at the
highest frequency being considered, and that this lightweight structure
could be controlled in standard observing conditions.}

To achieve aperture efficiency of 50\% at 10\,GHz with standard
horn feeds \citep{olver}, the unweighted
small-scale reflector accuracy was required to be better than 1.5-mm
r.m.s. This was achieved by combining theodolite
measurements of the completed antennas with finite-element analysis
for the worst-case load conditions. Follow-up measurements with radio
holography \citep{scott} using a number of geostationary satellite
beacons around 11.7\,GHz confirmed these results.

The antennas achieve blind pointing accuracy of 25\arcsec\ r.m.s under
all but the most extreme operating conditions, with a jitter of no
more than 5\arcsec\ over time scales of seconds. This is achieved at
tracking rates up to 0.05\degr/sec, across the elevation range. Below
2\,GHz the impact on measurements across the antenna beam is no more
than 1\% , and increases to a maximum of 5.5\% at 10\,GHz. Optical
pointing results show that referenced pointing \citep{rupen}, which
involves regularly checking the pointing against a pointing
calibration source near the field being observed, may allow accuracies
of better than 10\arcsec\ to be achieved, which would reduce the
impact to 2\%.
The focal ratio was chosen to be 0.38, which is close to optimal for
single pixel feeds using a conical horn feed \citep{rudge}.

\begin{figure}
\includegraphics[width=\columnwidth]{figs/f3.eps}
\caption{Two of the KAT-7 dishes viewed side-on. Note the lightning rods at the apex and near the focus, the backing structure of steel beams with circular holes, the small counterweight on the right and the sun shield on the left hand side of the pedestal. }
\label{fig:CAD}
\end{figure}



\begin{table}
\caption{Dish specification}
\begin{tabular}{lc}
\textbf{Parameter} & \textbf{Value}\\
\hline
Pointing Accuracy & 25\arcsec \\
Surface Accuracy & 1.5~mm r.m.s.~(spec.)\\
 & 1.0~mm r.m.s.~(goal) \\
Specified Upper Frequency Limit & 10 GHz \\
Wind (Operational)  & 36 km/h \\
Wind (Marginal Operation) km/h & 45 km/h \\
Wind (Drive to Stow) & 55 km/h \\
Wind (Survival)  & 160 km/h \\
Azimuth Rotation slew speed  & 2\degr /s \\
Azimuth limits &  $-$175\degr, +285 \degr \\
Elevation slew speed & 1\degr /s \\
Elevation limits & 0\degr, 90\degr \\
Diameter &  12 m \\
Focal ratio f/D & 0.38 \\
Lowest Natural Frequency & 3 Hz \\
Feed/Cryo Mass  & 75 kg \\
Mount Type & Alt--Az Prime Focus \\
\end{tabular}
\label{table:ant}
\end{table}

The picture of two dishes is shown in figure~\ref{fig:CAD} and dish
specifications are shown in table~\ref{table:ant}.  \textcolor{blue}{
  Primary beam measurements at L-band were initially done for total
  intensity mapping by raster scans across very bright radio sources
  (in particular Hercules~A, Orion~A, Taurus~A and Virgo~A). These
  were found to be consistent with modelling done by EMSS, the company
  doing the receiver systems and optical design. Later the beam shape
  was measured more accurately using a full polarization
  holography-like technique on the same bright sources (with some
  dishes tracking while others made spiral scans around the
  source). The broad features were the same, but the spiral scan mode
  showed details of the secondary lobes and the instrumental
  polarization in the primary beam.}

 \subsubsection{Why composites?}

 There are three main possible choices for a large dish antenna; a
 conventional steel structure based on panels, a lighter but possibly
 more expensive aluminium structure with panels, or a composite dish
 with metal backing for rigidity. For KAT-7 the choice was made for a
 composite for the following reasons:

\begin{itemize}
\item The dish front surface is constructed as a single unit, which
  gives it inherent stiffness (more efficient structure), something
  that is lost when the dish is constructed from loose panels.  This
  is an important aspect which reduces the mass while keeping the
  stiffness of the dish.
\item When constructed as a single unit there are no discontinuities
  in the reflective surface.
\item When constructed as a single unit there is no need to set up and
  align panels on a backing structure that is off the ground, which
  could prove to be time consuming and would be required for every
  dish.  Setting up of the mould is only required once, but must be
  done accurately.
\item Combining the composite dish surface with the steel rib and web
  backing gives a cost-effective solution.
\item An infusion process can be used, which is a tried-and-tested
  technology on such large structures.
\item Tooling is relatively inexpensive.

\end{itemize}

\textcolor{blue}{ It is also very convenient that the thermal
  expansion coefficients of the composite dish front and the steel
  backing are very close to each other, giving low thermal
  loading. Thermal loading is potentially a large problem for pointing
  and surface accuracy, given the large temperature changes in the
  semi-desert climate of the Karoo.} 
In addition the lightweight
nature of the dish meant that the counterweight needed would be light
and that the motors could be low powered and still achieve good slew
speeds.  \textcolor{blue}{ The main drawbacks for a composite laminate
  dish are that the mould accuracy must be higher than that of the
  desired surface and that the layers of the laminate must be done in
  a balanced and symmetrical way to reduce the inherently anisotropic
  nature of the fibre mats. For balance any laminae must be placed
  symmetrically around a centre line, and for symmetry any laminae at
  layer (\textit{n}) must be have an identical ply at layer
  (\textit{-n}) from the middle of the layers. For example a minimal
  set would be weave layers in the sequence (0,90), (45,-45), (45,-45)
  and (0,90)\degr\ . This symmetry is straightforward for a
  circularly symmetrical dish, but more complicated for other shapes.}
