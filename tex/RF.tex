\section{RF and IF chain}
\label{sec:RF}

\subsection{Feed}

\noindent
The feed provides native horizontal and vertical linear
polarization. It was decided not to convert to circular polarization
in the receiver. To add a polarizer would have increased the
complexity of the receiver, which was considered a risk. Adding the
required components (90\degr\ hybrid and isolators preceding the
low-noise amplifiers) inside the 80\,K cryostat with a single stage
Stirling cooler would increase system noise by at least 5\%.  A linear
polarization response could be calibrated sufficiently well to meet
the 30-dB linear and circular polarization purity required for
e.g.~measurements of Zeeman splitting; \textcolor{blue}{making
  a feed for a large fractional bandwidth (\textgreater~30\%) while
  retaining accurate circular polarization is difficult.}

\subsection {RFE Architecture}

\noindent
The Radio frequency Front End (RFE) architecture of KAT-7 is shown in 
Figure~\ref{fig:block}.
The design was driven by the following design decisions:

\begin{itemize}
\item Reduce complexity of the RFE at the antenna focus to reduce
  self-generated RFI to the maximum extent possible, in preference to
  mitigating RFI later.
\item Reduce complexity of the RFE at the antenna location and move
  the down-converter and clock distribution to the same location as
  the digitiser.
\item Test the concept of mitigating radio frequency interference
  (RFI) by moving the digitiser away from the antenna location to a
  shielded container located about 6\,km away, with the Losberg
  mountain providing a very high order of RFI shielding.
\end{itemize}  



\begin{figure*}
\includegraphics[width=\textwidth]{figs/f4.eps}
\caption{Signal chain from receiver to Digital Back End(DBE). Signals
  arrive at the horn and go through the OrthoMode Transducer (OMT) to
  the Low Noise Amplifier (LNA). Calibration from a noise generator
  can be injected. After further amplification and filtering they
  descend to the pedestal where they are again amplified. The signal
  level is adjusted by attenuators and the signals is then filtered
  and used to modulate an optical signal. This is transmitted over
  optical fibre and converted back to an RF signal in the Losberg
  facility. This then is mixed with local oscillators and digitally
  sampled for the Digital Back End}
\label{fig:block}
\end{figure*}



\begin{figure*}
\includegraphics[width=\textwidth]{figs/f5.eps}
\caption{An illustration of the down-conversion and sampling
  procedure. On the left side is the signal chain (top to bottom) with
  bandpass filters BF1-BF3, local oscillators LO1 and LO2 and the
  Analogue to Digital Converter (ADC). $f_{LO1}$ and $f_{LO2}$
  represent frequencies of the local oscillators. On the right are the
  spectra at the various stages. The top spectrum shows the signal as
  it arrives at Losberg, the middle spectrum shows the signal (and its
  alias) after mixing with LO1, while the bottom spectrum show the
  signal (and alias) after mixing with LO2, just before it is sampled
  by the ADC\@. N1 to N5 are the Nyquist zones. }
\label{fig:mixing}
\end{figure*}


 
\begin{figure*}
\includegraphics[width=\textwidth]{figs/f6.eps}
\caption{Detail of the last stage of the down-conversion procedure,
  showing the last (fixed) local oscillator and possible aliasing. The
  bandpass filter (BPF3) is designed to reduce that aliasing to a
  minimum before it enters the ADC, so that it does not appear in the
  sampled data. N1-N3 represent different Nyquist zones}
\label{fig:detail}
\end{figure*}

It is a double heterodyne system involving fixed and variable
local-oscillator (LO) signals that translate an RF (in the frequency
band of 1.2--1.95\,GHz) to an IF signal (in the band of 72--328\,MHz)
compatible with the digital samplers.  The first local oscillator
(LO1) can be set in the range 5.5--6.1\,GHz while the second (LO2) is
fixed at 4.0\,GHz.  The mixing scheme is shown in
Figure~\ref{fig:mixing}.  Figure~\ref{fig:detail} gives details of the
last stage of down-conversion and suppression of aliasing. `BPF's are
bandpass filters and `ADC's are analogue to digital converters.





\subsection{Noise Diodes}

%\textbf{XXXX SAY WHY WE HAVE A COUPLER *AND* A PIN XXXX}

\noindent
Each receiver includes two stabilized noise diodes to permit online
monitoring of system noise and receiver gain. The period and the duty
cycle of the noise system can be software selected, as well as having
a simple on/off mode. The signal can either be added to the received
(`sky') signal via an antenna that is integrated into the feed horn
(`the pin') or via 3-dB couplers that are installed between the
OrthoMode Transducer (OMT) ports and the low noise amplifiers (`the
coupler').  The `pin' signal level is approximately equivalent to the
system temperature (${T_{sys}}$) while the `coupler' level is
approximately $\frac{T_{sys}}{10}$. Using the `coupler' it is possible
to calibrate total-power measurements to an absolute scale with
accuracy better than 10\%.  The `pin' injection was designed to allow
delay and phase calibration between the orthogonal linear polarization
channels while the antennas were operating in single-dish mode, and
has been rarely used since the array was functional.  The receivers
were designed to permit observations of the Sun. However, no `high'
noise diode is provided to allow such observations to be calibrated.


\subsection{Down-conversion and Digitisation}

\noindent
To reduce the risk of RFI at the antennas, the analogue RF signal is
transported via fibre-optic cables to a central facility
\textcolor{blue}{behind Losberg mountain}. The signals are then
down-converted using a common LO, after which they are sampled in
baseband (i.e.~first Nyquist zone). The nominal signal is sampled with
3 bits (r.m.s.~$\approx 3$ levels) using 8-bit samplers. Together with
the 27\,dB of headroom to the 1-dB compression point provided by the
end-to-end analogue line-up, this provides the high dynamic range that
is required to limit the response to strong interfering signals. Even
on occasions when navigation satellites traverse the antenna beam,
gain compression remains at or below the 1\% level so intermodulation
products remain undetectable.

The analogue chain was designed to limit instrumental phase
instability to less than $8\degr$ r.m.s.~over 30-minute time scales. A
GPS-disciplined Rubidium maser clock is employed as frequency
reference to which the LOs are phase-locked. This allows the system to
achieve adequate phase stability to limit VLBI coherence loss to
20\% at frequencies up to 2\,GHz.
