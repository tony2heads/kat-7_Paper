\section{Correlator}
\label{sec:corr}

\noindent
The correlator modes are noted in Table~\ref{table:modes}.  The
correlator is based on the so-called `FX' architecture (Fourier
transform `F' followed by Cross-correlation `X') with Hann-windowed
polyphase filter banks (PFBs).

The spectral-line modes are limited to a single band per observation,
which is a minor inconvenience for observations of OH maser lines. The
correlator always computes \textcolor{blue}{all four complex
  polarization products} for all baselines (including
autocorrelations) since, with linearly-polarized feeds, total-power
measurements are easily corrupted by linear polarization and therefore
polarization calibration is required to achieve the highest possible
fidelity total-power imaging.

The correlator performs all internal computations with sufficient
numerical precision to support a spectral dynamic range of 43\,dB.

The Digital Back-End (DBE) also provides a coherently-summed
beamformer in parallel with a wide-bandwidth correlator. The
beamformer voltages are channelized to coincide with the correlator's
390\,kHz-wide channels. This correlator output is used by the Science
Processing Team's online pipeline to keep the
array phased-up during the course of the observation. The Science
Processor also provides the facility to re-synthesize and re-sample the
channelized voltages to the bandwidths required for the observation.


\begin{figure*}
\includegraphics[width=\textwidth]{figs/f7.eps}
\caption{The KAT-7 Correlator. The signal chain (in black) for two of
  the telescopes is shown, as well of the control and monitoring (in
  red). The F-engines do the Fourier transform and X-engines do the
  correlations. For $n$ antennas there are $\frac{n(n+1)}{2}$
  correlation products if you include all autocorrelations.}
\label{fig:correlatorbd}
\end{figure*}


The beamformer's output may be used for VLBI recording or for
performing coherent de-dispersion of pulsar signals.
Figure~\ref{fig:correlatorbd} shows a simplified block diagram of
KAT-7's prototype correlator, consisting of three main components:

\begin{itemize}
\item The \textit{F-engines} which channelise the incoming data
  streams into spectral components.
\item The \textit{X-engines} which multiply and accumulate every
  product pair.
\item A commodity, off-the-shelf network switch to interconnect these
  boards.
\end{itemize}


KAT-7 required additional features to turn the basic correlator into a
more fully-featured digital back-end, most notably beamforming (for
tied-array operation) as well as additional operations in the
correlator itself such as fringe rotation and delay compensation, fast
readout speeds and a standardised output protocol. Hardware changes
were also required: The performance of the existing digitisers within
the CASPER\footnote{\url{https://casper.berkeley.edu}} collaboration
were found to be inadequate for KAT-7, and there was a desire for
KAT-7 to standardise on a single, general-purpose hardware processing
platform that could both digitise and process the signals.

In its current form, the KAT-7 DBE consists of 16 ROACH boards, eight of
which are configured as F-engines and the remaining eight as X-engines. It
is a real-time, full-Stokes, asynchronous packetized design with 16
inputs, processing 400\,MHz of RF bandwidth (only 256\,MHz of which
contains useful analogue signal because of the anti-aliasing filters
introduced in the RF chain (see Figure~\ref{fig:detail}).
ROACH boards can optionally host ADCs for digitising analogue signals
and the F-engines in KAT-7 are equipped with KAT ADCs to digitise the
down-converted IF signal at 800\,MSample/s with an 8-bit resolution. A
20-port 10\,GbE network switch is used to interconnect all the
processing boards.

While the F-engines and their ADCs are synchronously clocked from a
GPS-disciplined rubidium common clock source, the X-engines are purely
compute nodes that operate from their own, asynchronous,
clocks. Packets are timestamped in the F-engines and these headers are
interrogated by subsequent processors for data identification and
re-alignment. These \textcolor{blue}{X-engines} could easily be
replaced by ordinary CPUs and GPUs when technology advances enable
them to cope with the data rates presented.

The fringe rotation and delay compensation were implemented within the
F-engines by employing a combination of time-domain and
frequency-domain processing, and the system currently allows for
phased-tracked wideband (continuum) and spectral-line observations. It
does not allow online Doppler tracking.

Frequency-domain beamforming is implemented by tapping a copy of the
channelised data destined for the X-engines, and presenting it to a
co-located \textit{B-engine} which performs beam steering and
summation. The existing F-engines are thus shared by the X- and
B-engines. Development time and risk were reduced by leveraging the
existing delay-compensation software. This B-engine is not resource
hungry, needing only a single complex multiplier and adder to sum the
already serialised data, along with a lookup table for the steering
coefficients. This makes it possible to create additional beams with
modest incremental hardware costs.

All systems are being designed to be directly transportable and
scalable up to larger arrays such as MeerKAT and the SKA.
