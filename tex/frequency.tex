\section{Choice of Frequency}
\label{sec:freq}
%based on stuff by Adriaan

\noindent
One particular scientific niche that was identified for a small-scale
interferometer in the southern hemisphere was to make high-sensitivity
observations of low surface-brightness emission from extended neutral
hydrogen in the nearby Sculptor cluster of galaxies. The
interferometer's response at 1.4\,GHz needed to be sufficient to
resolve structures on angular scales up to 24\arcmin\ (one third of
the antenna primary beam). This set the shortest baseline length at
30\,m. The positions of the six remaining antennas were selected to
optimize the interferometer response for 4-hour observations,
resulting in a randomized distribution with longest baseline of
180\,m.

The receiver frequency range was chosen to be 1200--1950\,MHz in order
to avoid potential interference from terrestrial GSM and
aeroplanes. The 1.63:1 bandwidth ratio was still feasible with
corrugated horn feeds. This frequency range also gave the possibility
of joining in with standard 18-cm VLBI observations of OH masers and
continuum sources.
